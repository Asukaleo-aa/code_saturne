%-------------------------------------------------------------------------------
\section{Eddy viscosity Models (\emph{EVM})}
In this section eddy viscosity hypothesis is made which states that the Reynolds stress tensor
is aligned with the rate of strain $\tens{S}$:
%
\begin{equation}
\tens{R} = \dfrac{2}{3}k \tens{Id} - 2 \mu_T \deviator{\tens{S}}
\end{equation} 

%-------------------------------------------------------------------------------
\subsection{Equations for the variables $k$ and $\varepsilon$ (standard $k-\varepsilon$ model)}

\begin{equation}
\left\{
\begin{array}{r c l}
\displaystyle\frac{\partial }{\partial t}(\rho k)+\dive\left[ \rho \vect{u}%
\,k- \left( \mu +\frac{\mu _{t}}{\sigma _{k}} \right)\grad{k}\right] 
&=&
\mathcal{P}+\mathcal{G}-\rho \varepsilon +\Gamma k^{in}+ST_{k}, \\
\displaystyle\frac{\partial }{\partial t}(\rho \varepsilon )+\dive\left[
\rho \vect{u}\,\varepsilon -(\mu +\frac{\mu _{t}}{\sigma _{\varepsilon }})%
\grad{\varepsilon}\right] &=&C_{\varepsilon _{1}}\frac{\varepsilon }{k}\left[
\mathcal{P}+(1-C_{\varepsilon _{3}})\mathcal{G}\right] -\rho C_{\varepsilon
_{2}}\frac{\varepsilon ^{2}}{k}+\Gamma \varepsilon ^{in}+ST_{\varepsilon },%
\end{array}%
\right.
\end{equation}
\nomenclature[rk]{$k$}{turbulent kinetic energy \nomunit{$m^{2}.s^{-2}$}}
\nomenclature[gepslion]{$ \varepsilon $}{turbulent kinetic energy dissipation \nomunit{$m^{2}.s^{-3}$}}
\nomenclature[rproduction]{$\mathcal{P}$}{turbulent kinetic energy production \nomunit{$kg.m^{-1}.s^{-3}$}}
where $\mathcal{P}$ is the production term created by mean shear:
%
\begin{equation}
\begin{array}{rcl}
\mathcal{P} & = & \displaystyle -\rho \tens{R} : \gradt \, \vect{u}
= -\left[-2 \mu_t \deviator{\tens{S}}%
+ \frac{2}{3}\rho k \tens{1}\right] : \tens{S}, \\
& = & \displaystyle \mu_t \left[ 2\left(\frac{\partial u}{\partial x}%
\right)^2+ 2\left(\frac{\partial v}{\partial y}\right)^2+ 2\left(\frac{%
\partial w}{\partial z}\right)^2+ \left(\frac{\partial u}{\partial y}+\frac{%
\partial v}{\partial x}\right)^2+ \left(\frac{\partial u}{\partial z}+\frac{%
\partial w}{\partial x}\right)^2+ \left(\frac{\partial v}{\partial z}+\frac{%
\partial w}{\partial y}\right)^2 \right] \\
&&-\frac{2}{3}\mu_t \left( \dive\vect{u} \right)^2-\frac{2}{3}
\rho k \dive \left( \vect{u} \right),
\end{array}
\end{equation}
\nomenclature[rbuoyancy]{$\mathcal{G}$}{turbulent kinetic energy buoyancy term \nomunit{$kg.m^{-1}.s^{-3}$}}
and
$\mathcal{G}$ is the production term created by gravity effects: 
\begin{equation}
\displaystyle \mathcal{G}= \frac{1}{\rho}\frac{\mu_t}{\sigma_t} \grad \rho \cdot \vect{g}.
\end{equation}

The turbulent viscosity is: 
\begin{equation}
\displaystyle \mu_t=\rho C_\mu\frac{k^2}{%
\varepsilon}.
\end{equation}
\nomenclature[rstk]{$ST_{k}$}{additional turbulent kinetic energy source term \nomunit{$kg.m^{-1}.s^{-3}$}}
\nomenclature[rstepsilon]{$ST_{\varepsilon}$}{additional turbulent dissipation source term \nomunit{$kg.m^{-1}.s^{-4}$}}
$ST_{\varphi }$ ($\varphi =k$ or $\varepsilon $) stands for the additional
source terms prescribed by the user (in rare cases only).

The constants of the model are given in the Table (\ref{tab:k_epsilon_constants}):
\begin{table}[htp]
\centering
\begin{tabular}{p{0,8cm}|p{0,8cm}|p{0,8cm}|p{0,8cm}|p{0,8cm}}
$C_\mu$ & $C_{\varepsilon_1}$ & $C_{\varepsilon_2}$ & $\sigma_k$ & $%
\sigma_\varepsilon$ \\ \hline
$0,09$ & $1,44$ & $1,92$ & $1$ & $1,3$ 
\end{tabular}%
\caption{Standard $k-\varepsilon$ model constants \cite{Launder:1972}.\label{tab:k_epsilon_constants}}
\end{table}
\nomenclature[rcmu]{$C_\mu$}{eddy viscosity constant}
\nomenclature[rcepsilon1]{$C_{\varepsilon_1}$}{constant of the standard $k-\varepsilon$ model}
\nomenclature[rcepsilon2]{$C_{\varepsilon_2}$}{constant of the standard $k-\varepsilon$ model}
\nomenclature[rcepsilon3]{$C_{\varepsilon_3}$}{constant of the standard $k-\varepsilon$ model depending on the buoyancy term}

$C_{\varepsilon_3}=0$ if $\mathcal{G}\geqslant0$ (unstable stratification)
and $C_{\varepsilon_3}=1$ if $\mathcal{G}\leqslant0$ (stable stratification).

%-------------------------------------------------------------------------------
\subsection{$k-\varepsilon$ with Linear Production (\emph{LP}) model}
work in progress

%-------------------------------------------------------------------------------
\subsection{$k-\omega$ \emph{SST} model}
work in progress

%-------------------------------------------------------------------------------
\subsection{Spalart-Allmaras model}
work in progress

%-------------------------------------------------------------------------------
\section{Differential Reynolds Stress Models (\emph{DRSM})}

%-------------------------------------------------------------------------------
\subsection{Equations for the Reynolds stress tensor components $R_{ij}$ 
and $\varepsilon$ (\emph{LRR} model)}
%
\nomenclature[rrt2]{$\tens{R}$}{Reynolds stress tensor \nomunit{$m^{2}.s^{-2}$}}
\nomenclature[rrij]{$R_{ij}$}{componant $ij$ of the Reynolds stress tensor \nomunit{$m^{2}.s^{-2}$}}

\begin{equation}
\left\{
\begin{array}{rcll}
\displaystyle\frac{\partial }{\partial t} \left(\rho \tens{R} \right)
+\divt \left( \tens{R} \otimes \rho \vect{u}-\mu \gradtt \tens{R} \right) 
& = & \tens{\mathcal{P}} + \tens{G}+ \tens{\Phi} + \tens{\mathit{d}}
-\rho \tens{\varepsilon} & \displaystyle+\Gamma \tens{R}^{in}+\tens{ST}_{R_{ij}},
\\
\displaystyle\frac{\partial }{\partial t}(\rho \varepsilon )
+\dive\left(\rho \vect{u}\,\varepsilon -\mu \grad{\varepsilon}\right) 
& = & \displaystyle \mathit{{d}+C_{\varepsilon _{1}}\frac{\varepsilon }{k}\left[ \mathcal{P}%
+G_{\varepsilon }\right] -\rho C_{\varepsilon _{2}}\frac{\varepsilon ^{2}}{k}} 
& \displaystyle+\Gamma \varepsilon ^{in}+ST_{\varepsilon },
\end{array}%
\right.
\end{equation}
\nomenclature[rproductiont2]{$\tens{\mathcal{P}}$}{turbulent production tensor \nomunit{$kg.m^{-1}.s^{-3}$}}
\nomenclature[rbuoyancyt2]{$\tens{\mathcal{G}}$}{turbulent buoyancy production tensor \nomunit{$kg.m^{-1}.s^{-3}$}}
where
$\tens{\mathcal{P}}$ stands for the turbulence production tensor associated
with mean flow strain-rate and $\tens{\mathcal{G}}$ is stands for the
production- tensor associated with buoyancy effects:
\begin{equation}
\begin{array}{r c l}
\displaystyle \tens{ \mathcal{P}} & = & \displaystyle-\rho \left[ \tens{R} \cdot \gradt \, \vect{u} 
+ \gradt \, \vect{u}  \cdot \tens{R}\right], \\
\tens{ \mathcal{G}} & = &
\displaystyle \frac{3}{2}\frac{C_{\mu }}{\sigma _{t}}
\frac{k}{\varepsilon }
\left[\vect{r} \otimes \vect{g} +\vect{g} \otimes \vect{r}  \right].
\end{array}
\end{equation}
where $ \vect{r} \equiv \tens{R} \cdot \grad  \rho$ and 
$G_{\varepsilon }= \Max \left(0, \, \frac{1}{2}\trace \tens{\mathcal{G}}\right)$.
\nomenclature[rrt1]{$\vect{r}$}{vector of the Reynolds stress tensor times the density gradient}
\nomenclature[rbuoyancyepsilon]{$G_{\varepsilon }$}{turbulent buoyancy term for dissipation}

With these definition the following relations hold:
\begin{equation}
\begin{array}{r c l}
\displaystyle k &=&\frac{1}{2} \trace{\tens{R}}, \\
\mathcal{P} &=&\frac{1}{2} \trace \left( \tens{\mathcal{P}} \right) ,
\end{array}
\end{equation}

$\tens{\Phi}$ is the term representing pressure-velocity correlations:
\nomenclature[gphit2]{$\tens{\Phi}$}{pressure-velocity correlation tensor \nomunit{$kg.s^{-3}$}}
\begin{equation}
\displaystyle \tens{\Phi} = \tens{\phi_{1}}+ \tens{\phi_{2}}+ \tens{\phi_{3}}+ \tens{\phi_{w}},
\end{equation}%
%
\begin{equation}
\begin{array}{r c l}
\tens{\phi_{1}} &=& \displaystyle -\rho \,C_{1}\frac{\varepsilon }{k}%
\deviator{\tens{R}}, \\
\tens{\phi_{2}} &=& -\rho \,C_{2} 
\deviator{\tens{\mathcal{P}}}, \\
\tens{\phi_{3}} &=& -C_{3} \deviator{ \tens{G} }.
\end{array}
\end{equation}

The term $\tens{\phi_{w}}$ is called ``wall echo term'' (by default, it is not
accounted for: see \fort{turrij} \ref{ap:turrij}).

The dissipation term, $\tens{\varepsilon}$ , is considered isotropic:
\nomenclature[gepsilont2]{$\tens{\varepsilon}$}{turbulent kinetic energy dissipation tensor \nomunit{$m^{2}.s^{-3}$}}
\begin{equation}
\displaystyle \tens{\varepsilon}=\frac{2}{3}\ \varepsilon \tens{1}.
\end{equation}

The turbulent diffusion terms are:
\begin{equation}
\begin{array}{r c l}
\tens{d} & = & C_{S} \displaystyle \divt \left( \rho \frac{k}{\varepsilon }%
\tens{R} \cdot \gradtt \tens{R} \right), \\
d & = & C_{\varepsilon }\displaystyle \dive \left( \rho \frac{k}{\varepsilon} 
\tens{R} \cdot \grad \varepsilon \right).
\end{array}
\end{equation}

In the rare event of mass sources, $\Gamma R_{ij}^{in}$ and $\Gamma
\varepsilon ^{i}$ are the corresponding injection terms. $ST_{R_{ij}}$ and $%
ST_{\varepsilon }$ are also rarely used additional source terms that can
prescribed by the user.

\begin{table}[!htp]
\begin{center}
\begin{tabular}{p{0,8cm}|p{0,8cm}|p{0,8cm}|p{0,8cm}|p{0,8cm}|p{0,8cm}|p{0,8cm}|p{0,8cm}|p{0,8cm}|p{0,8cm}}
$C_\mu$ & $C_{\varepsilon}$ & $C_{\varepsilon_1}$ & $C_{\varepsilon_2}$ & $%
C_1$ & $C_2$ & $C_3$ & $C_S$ & $C^{\prime}_1$ & $C^{\prime}_2$ \\ \hline
$0,09$ & $0,18$ & $1,44$ & $1,92$ & $1,8$ & $0,6$ & $0,55$ & $0.22$ & $0,5$
& $0,3$ 
\end{tabular}
\end{center}
\caption{Model constants of the \emph{LRR}??? $R_{ij}-\varepsilon$ model \cite{Launder:????}.}
\end{table}

%-------------------------------------------------------------------------------
\section{Large-Eddy Simulation (\emph{LES})}

%-------------------------------------------------------------------------------
\subsection{Standard Smagorinsky model}
\subsection{Dynamic Smagorinsky model}
\subsection{WALE model}

\begin{description}
 \item[Smagorinsky model] 
\begin{equation}
\mu_{t}=\rho \, \left( C_{s}\,\overline{\Delta } \right)^{2}
\sqrt{2\overline{\tens{S}} \,: \, \overline{\tens{S}}},
\end{equation}%
\nomenclature[odotproductdouble]{$:$}{double dot product}
%
where $\overline{\tens{S}}$ the filtered strain rate tensor components:

\begin{equation}
\overline{\tens{S}}= \symmetric{\overline{\tens{S}}} =
\frac{1}{2} \left[ \gradt \vect{\overline{u}} + \transpose{\left( \gradt \, \vect{\overline{u}} \right)}
\right].
\end{equation}
%
\nomenclature[osymmetric]{$\symmetric{ \left(\tens{.}\right)}$}{symmetric part of a tensor}
%
Here, $\overline{u_{i}}$ stands for the $i^{th}$ resolved velocity component
\footnote{%
In the case of implicit filtering, the discretization in space introduces a
spectral low pass filter: only the structures larger that twice the size of
the cells are accounted for. Those structures are called ''the resolved
scales'', whereas the rest, $u_{i}-\overline{u_{i}}$, is referred to as
''unresolved scales'' or ''sub-grid scales''. The influence of the
unresolved scales on the resolved scales have to be modelled.}. 

$C$ is the Smagorinsky constant. Its theoretical value is $0.18$ for
homogeneous isotropic turbulence, but the value $0.065$ is classic for
channel flow. 

$\overline{\Delta }$ is the filter width associated with the finite volume
formulation (implicit filtering which corresponds to the integration over a
cell). The value recommended for hexahedral cells is: $\overline{\Delta }
=2 \norm{\vol{\celli}}^{\frac{1}{3}}$where $\norm{\vol{\celli}}$ is the volume of the cell $\celli$.

\item[Classic dynamic model]
A second filter is introduced: it is an explicit filter with a
characteristic width $\widetilde{\Delta }$ superior to that of the implicit
filter ($\overline{\Delta }$). If $\varia$ is a discrete variable defined over
the computational domain, the variable obtained after applying the explicit
filter to $\varia$ is noted $\tilde{\varia}$. Moreover, with:

\begin{equation}
\begin{array}{ r c l}
\tens{L} & = &\widetilde{\overline{\vect{u}} \otimes \overline{\vect{u}}}
-\widetilde{\overline{\vect{u}}} \otimes \widetilde{\overline{ \vect{u}}}, \\
\tens{\tau} & = & \overline{ \vect{u} \otimes \vect{u}}-\overline{\vect{u}} \otimes \overline{ \vect{u}}, \\
\tens{T} &= &\widetilde{\overline{ \vect{u} \otimes \vect{u}}}-\widetilde{\overline{\vect{u}}} \otimes
\widetilde{\overline{ \vect{u}}},
\end{array}
\end{equation}
Germano identity reads:
\begin{equation}
\tens{L} = \tens{T}-\widetilde{\tens{\tau}}.
\end{equation}

Both dynamic models described herafter rely on the estimation of the tensors
$\tens{T}$ and $\tens{\tau}$ as functions of the filter widths and of the
strain rate tensor (Smagorinsky model). The following modelling is adopted%
\footnote{$\delta_{ij}$ stands for the Kroeneker symbol.}:

\begin{equation}
\begin{array}{ r c l}
T_{ij}-\frac{1}{3}\trace \tens{T} \delta_{ij} &=& -2 C \widetilde{\Delta}^2 |\widetilde{%
\overline{D_{ij}}}| \widetilde{\overline{D_{ij}}}, \\
\tau_{ij}-\frac{1}{3} \tens{\tau } \delta_{ij} &=& -2 C^* \overline{\Delta}^2 |%
\overline{D_{ij}}| \overline{D_{ij}} ,
\end{array}
\end{equation}
where 
$\overline{u}$ stands for the ``implicit-filtered" value of a variable $u$
defined at the centres of the cells and $\overline{u}$ represents the
``explicit-filtered" value associated with the variable $u$. It follows that
the numerical computation of $L_{ij}$ is possible, since it requires the
explicit filtering to be applied to implicitly filtered variables only 
(\textit{i.e.} to the variables explicitly computed).

On the following assumption:

\begin{equation}
C = C^*,
\end{equation}
and assuming that $C^*$ is only slightly non-uniform, so that it can be
taken out of the explicit filtering operator, the following equation is
obtained:

\begin{equation}
\deviator{\tens{L}} =  C \left(
\tens{ \alpha}- \tens{\widetilde{\beta}} \right),
\end{equation}
with:
\begin{equation}
\begin{array}{rcl}
\alpha_{ij} &=& -2 \widetilde{\Delta}^2 |\widetilde{\overline{D_{ij}}}|
\widetilde{\overline{D_{ij}}} , \\
\beta_{ij} &=& -2 \overline{\Delta}^2 |\overline{D_{ij}}| \overline{D_{ij}}.
\end{array}
\end{equation}

Since we are left with six equations to determine one single variable, the
least square method is used\footnote{$L_{kk}$ has no effect for
incompressible flows.}. With:
\begin{equation}
\tens{E} = \tens{L}-C \left( \tens{\alpha} - \tens{\widetilde{\beta}} \right),
\end{equation}
the value for $C$ is obtained by solving the following equation:
\begin{equation}
\frac{\partial \tens{E} : \tens{E}}{\partial C} = 0.
\end{equation}

Finally:
\begin{equation}
C = \frac{ \tens{M} : \tens{L} }{ \tens{M} : \tens{M}},
\end{equation}
with
\begin{equation}
\tens{M} = \tens{\alpha} - \tens{\widetilde{\beta}}.
\end{equation}

This method allows to calculate the Smagorinsky "constant" dynamically at
each time step and at each cell. However, the value obtained for $C$ can be
subjected to strong variations. Hence, this approach is often restricted to
flows presenting one or more homogeneous directions (Homogeneous Isotropic
Turbulence, 2D flows presenting an homogeneous span-wise direction...):
indeed, in such cases, the model can be (and is) stabilized by replacing $C$
by an average value of $C$ computed over the homogeneous direction(s).

For a general case (without any homogeneous direction), a specific averaging
is introduced to stabilize the model: for any given cell of the mesh, the
averaged Smagorinsky constant is obtained as an average of $C$ over the
"extended neighbourhood" of the cell (the set of cells that share at least
one vertex with the cell considered). More precisely, the average value
(also denoted $C$ hereafter) is calculated as indicated below:

\begin{equation}
C = \frac{\widetilde{ \tens{M} : \tens{L}}} {\widetilde{ \tens{M} : \tens{M}}}
\end{equation}

\end{description}

%-------------------------------------------------------------------------------
\subsection{Dynamic Smagorinsky model}
work in progress
\subsection{WALE model}
work in progress

