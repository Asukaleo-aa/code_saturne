\section{System equations}
\label{sec:cavit:system_equations}
The cavitation model is based on an homogeneous mixture model. In this
model, the physical properties, the density $\rho$ and the dynamic
viscosity $\mu$, of the mixture depends on a resolved void fraction
$\alpha$ and constant reference properties $\rho_l, \, \mu_l$ for the
liquid phase and $\rho_v, \, \mu_v$ for the gas phase, following the
relations:
\begin{align}
%\begin{array}
\label{eq:cavit:eq_rho}\rho&= \alpha\rho_v+(1-\alpha)\rho_l, \\
\label{eq:cavit:eq_mu}\mu&= \alpha\mu_v+(1-\alpha)\mu_l.
%\end{array}
\end{align}
In this model, it is assumed that the mixture dynamic is ruled by the
classical incompressible Navier--Stokes equations:
\begin{equation}
\label{eq:cavit:mass}
\frac{\partial\rho}{\partial t}+\divs \left(\rho\vect{u} \right) = 0,
\end{equation}
\begin{equation}
\label{eq:cavit:qdm}
\dfrac{\partial }{\partial t}(\rho \vect{u})
+\divv \left( \vect{u}\otimes \rho \vect{u} \right)
= - \grad P + \divv  \tens{\tau}.
\end{equation}
Eq.~\eqref{eq:cavit:qdm} corresponds to the momentum equation in
Eq.~\eqref{eq:navier_stokes_conservative} where the volume source
terms have been removed for brevity and $\Gamma\equiv0$ for the mass
source term.

Using \eqref{eq:cavit:eq_rho}, the mass equation can be splitted into:
\begin{equation}
\label{eq:cavit:eq_vf0}
\der{(\alpha\rho_v)}{t}+\divs(\alpha\rho_v\vect{u}) = \Gamma_V,
\end{equation}
\begin{equation}
\label{eq:cavit:eq_lf0}
\der{\left((1-\alpha)\rho_l\right)}{t}+\divs\left((1-\alpha)\rho_l\vect{u}\right) = -\Gamma_V ,
\end{equation}
with $\Gamma_V$ representing the vaporization (resp. condensation) source (resp. sink)
term, appearing with an opposite sign in the liquid part of the
density budget.

Using the fact that the reference densities $\rho_v$ and $\rho_l$ are
constant, Eqs.~\eqref{eq:cavit:eq_vf0} and \eqref{eq:cavit:eq_lf0} can be easilly
written in the form:
\begin{equation}
\label{eq:cavit:eq_vf1}
\der{\alpha}{t} + \divs(\alpha\vect{u}) =
\dfrac{\Gamma_V}{\rho_v},
\end{equation}
%
\begin{equation}
\label{eq:cavit:eq_lf1}
\divs(\vect{u})=\Gamma_V\left(\frac{1}{\rho_v}-\frac{1}{\rho_l}\right).
\end{equation}
It is seen that the mass equation of the mixture Eq.~\eqref{eq:cavit:eq_rho}
has been splitted into two equations: one simple convection equation
which can be used to solve the void fraction and one equation
structurally similar to the one solved at the correction step of the
predictior-corrector solver of \CS (see \appendixname~\ref{ap:resopv}).
The global resolution scheme of the cavitation module is thus the
following:
\begin{enumerate}
\item Prediction of the velocity using Eq.~\eqref{eq:cavit:qdm} (see
\appendixname~\ref{ap:predvv}).
\item Correction of the velocity using Eq.~\eqref{eq:cavit:eq_lf1} (see
\appendixname~\ref{ap:resopv}).
\item Resolution of the void fraction using Eq.~\eqref{eq:cavit:eq_vf1}.
\item Update physcal properties of the mixture using
Eqs.~\eqref{eq:cavit:eq_rho} and \eqref{eq:cavit:eq_mu}.
\end{enumerate}
%
\section{Vaporization source term}
\label{sec:cavit:source_term}
In the cavitation module of \CS, the $\Gamma_V$ source term is
modeled using the Merkle model:
\begin{equation*}
\Gamma_v \left(\alpha, P\right) = m^+ + m^-,
\end{equation*}
 with:
\begin{equation}
m^+ = -\dfrac{C_{prod} \rho_l \min \left( P-P_V,0
\right)\alpha(1-\alpha)}{0.5\rho_lu_\infty^2t_\infty},\qquad
m^- = -\dfrac{C_{dest} \rho_v \max \left( P-P_V,0
\right)\alpha(1-\alpha)}{0.5\rho_lu_\infty^2t_\infty},
\end{equation}
and
$ C_{prod} = 1000, \, C_{dest} = 50 $
empirical constants, $t_\infty=l_\infty/u_\infty$ a reference time
scale and $P_V$ the reference saturation pressure. $l_\infty$,
$u_\infty$ and $P_V$ may be provided by the user.
%
\section{Time stepping}\label{sec:cavit:time_stepping}
At each time step $n$, the global resolution scheme follows successively
the steps 1 to 4 described above (\S
\ref{sec:cavit:system_equations}). In this resolution scheme, the physical
properties are updated at the end of the time step $n$.

\paragraph{Prediction step} The procedure is almost identical to the
classical one described at \appendixname~\ref{ap:predvv}. Only the
discretization of the time derivative at the l.h.s of
Eq.~\eqref{eq:cavit:qdm} is modified in order to take into account that the
mixture density is updated at the end of the time step. In the
cavitation algorithm, the time derivative at time step $n+1$ is
discretized by:
\begin{equation*}
\der{}{
t}(\rho\vect{u})\simeq \frac{\rho^n\widetilde{\vect{u}}^{n+1}-\rho^{n-1}\vect{u}^n}{\Delta t^n},
\end{equation*}
%
with $\widetilde{\vect{u}}^{n+1}$ the predicted velocity at time step
$n+1$.
%
\paragraph{Correction step}
With the Merkle model described above (\S
\ref{sec:cavit:source_term}), the correction step equation of the
cavitation model writes:
\begin{equation*}
\divs\left(\frac{\Delta t^n}{\rho}\grad\delta P\right) =
\divs(\widetilde{u})-\Gamma_V(\alpha,P).
\end{equation*}
In this equation, the pressure in the cavitation source term is taken
implicit while the void fraction is explicit:
\begin{equation*}
\divs\left(\frac{\Delta t^n}{\rho}\grad(\delta P)^{n+1}\right) =
\divs(\widetilde{u}^{n+1})-\Gamma_V(\alpha^n,P^{n+1})\left(\frac{1}{\rho_v}-\frac{1}{\rho_l}\right).
\end{equation*}
%
\paragraph{Void fraction resolution}
The time discretization of Eq.~\eqref{eq:cavit:eq_lf1} is:
\begin{equation}\label{eq:cavit:eq_vf_dis}
\frac{\alpha^{n+1}-\alpha^n}{\Delta t^n}+\divs(\alpha^{n+\theta}\vect{u}^{n+\theta}) =
\frac{1}{\rho_v}\Gamma_V(\alpha^n,P^{n+1}).
\end{equation}
In this equation, the cavitation source term is discretized with the
time scheme as the one used at the correction step in order to
ensure the exact mass conservation (it is recalled that the void
faction transport equation and correction step are two part of the
mixture mass conservation equation, Eq.~\eqref{eq:cavit:mass}.

%
%\section{Space discretization}
