%-------------------------------------------------------------------------------

% This file is part of Code_Saturne, a general-purpose CFD tool.
%
% Copyright (C) 1998-2013 EDF S.A.
%
% This program is free software; you can redistribute it and/or modify it under
% the terms of the GNU General Public License as published by the Free Software
% Foundation; either version 2 of the License, or (at your option) any later
% version.
%
% This program is distributed in the hope that it will be useful, but WITHOUT
% ANY WARRANTY; without even the implied warranty of MERCHANTABILITY or FITNESS
% FOR A PARTICULAR PURPOSE.  See the GNU General Public License for more
% details.
%
% You should have received a copy of the GNU General Public License along with
% this program; if not, write to the Free Software Foundation, Inc., 51 Franklin
% Street, Fifth Floor, Boston, MA 02110-1301, USA.

%-------------------------------------------------------------------------------

%%%%%%%%%%%%%%%%%%%%%%%%%%%%%%%%%%%%%%%%%%%%%%%%%%%%%%%%%%%%%%%%%%%%%%
% SYNTH�SE
\section{Introduction}

\CS is a system designed to solve the Navier-Stokes
equations in the cases of 2D, 2D axisymmetric or 3D flows. Its main module is
designed for the simulation of flows which may be steady or
unsteady, laminar or turbulent, incompressible or potentially dilatable,
isothermal or not. Scalars and turbulent fluctuations of scalars can be taken into
account. The code includes specific modules, referred to as ``specific physics'',
for the treatment of lagrangian particle tracking, semi-transparent radiative transfer,
gas, pulverized coal and heavy fuel oil combustion,
electricity effects (Joule effect and electric arcs) and compressible flows.
\CS relies on a finite volume discretization and allows the use of
various mesh types which may be hybrid (containing several kinds of
elements) and may have structural non-conformities (hanging nodes).

The present document is a tutorial for \CS version \verscs.
It presents five simple test cases and guides the future \CS user step by step
into the preparation and the computation of the cases.

The test case directories, containing the necessary meshes and data
are available in the \texttt{examples} directory.

This tutorial focuses on the procedure and the preparation of the \CS
computations. For more elements on the structure of the code and the definition
of the different variables, it is higly recommended to refer to the user
manual.

\CS is free software; you can redistribute it
and/or modify it under the terms of the GNU General Public License
as published by the Free Software Foundation; either version 2 of
the License, or (at your option) any later version.
\CS is distributed in the hope that it will be
useful, but WITHOUT ANY WARRANTY; without even the implied warranty
of MERCHANTABILITY or FITNESS FOR A PARTICULAR PURPOSE.  See the
GNU General Public License for more details.
%
%%%%%%%%%%%%%%%%%%%%%%%%%%%%%%%%%%%%%%%%%%%%%%%%%%%%%%%%%%%%%%%%%%%%%%
