%-------------------------------------------------------------------------------

% This file is part of Code_Saturne, a general-purpose CFD tool.
%
% Copyright (C) 1998-2011 EDF S.A.
%
% This program is free software; you can redistribute it and/or modify it under
% the terms of the GNU General Public License as published by the Free Software
% Foundation; either version 2 of the License, or (at your option) any later
% version.
%
% This program is distributed in the hope that it will be useful, but WITHOUT
% ANY WARRANTY; without even the implied warranty of MERCHANTABILITY or FITNESS
% FOR A PARTICULAR PURPOSE.  See the GNU General Public License for more
% details.
%
% You should have received a copy of the GNU General Public License along with
% this program; if not, write to the Free Software Foundation, Inc., 51 Franklin
% Street, Fifth Floor, Boston, MA 02110-1301, USA.

%-------------------------------------------------------------------------------

\programme{cpbase}

%%%%%%%%%%%%%%%%%%%%%%%%%%%%%%%%%%
%%%%%%%%%%%%%%%%%%%%%%%%%%%%%%%%%%
\section{Function}
%%%%%%%%%%%%%%%%%%%%%%%%%%%%%%%%%%
%%%%%%%%%%%%%%%%%%%%%%%%%%%%%%%%%%


Pulverised coal combustion is described ({\small excluding grid
burning}) allowing the use of mixture of coals ({\small or of coal and
biomass}) and a description of granulometry ({\small as many classes
of initial diameter as wished}). After a particle enters the furnace,
radiation increases its temperature.
\begin{enumerate}
  \item As particle's temperature increases, evaporation of free water
  ({\small if any}) begins. During evaporation, the vapor pressure
  gradient extract some water from the particle. The interfacial mass
  flux brings some mass, water and enthalpy ({\small computed for
  water vapour}) then latent heath is taken from the particle's
  enthalpy so the heating is slowed ({\small during evaporation, the
  water can't reach the boiling point}).
\item After drying is achevied, the temperature  reachs higher level, allowing pyrolysis to take place. The pyrolysis is described by two competitive reactions : the first one with a moderate activation energy is able to free peripherals atoms group from skeleton leaving to light gases and a big amount of char ; the second one with an higher activation enrgy is able to break links deeper in the skeleton leaving to heavier gases ({\small or tar}) and less char ({\small more porous}. So a complete description needs two set of three parameters ({\small two kinetics ones and a partitioning one}): \\
\\
\centerline{$Coal =(k01, T01)=> Y1. light volatiles + (1-Y1).char$}\\
\centerline{$Coal =(k02, T02)=> Y2. heavy volatiles + (1-Y2).char$}\\
\\ Where Y1, the partitionning ({\small or selectivity}) factor of the "{\em low temperature}" reaction is less than Y2, the "{\em high temperature}" one. A practical rule is to consider that the same hydrogen and oxygen can bring twice more carbon by the second reaction than by the first one.\\


When ultimate analysis are available both for coal and for char, it is
 releavant to check partioning coefficient ({\small $Y_{i}$}) and
 composition of volatiles matters ({\small mainly ratio of Carbon
 monoxide and C$/$H in the hydrocarbon fraction}) : assumptions on
 volatiles composition gives partitionning coefficients ; assumptions
 on $Y_{i}$ determine volatiles equivalent formulae. Pyrolisis
 interfacial mass flux brings energy of volatile gases ({\small
 computed at the particle's temperature}) in wich the formation of
 enthalpy of gaseous species differs from the coal one's, as a result
 the enthalpy for pyrolisis reaction ({\small the most ofen, moderate})
 is taken from particle energy.  \item After pyrolysis, when every
 volatiles are burnt, oxygen is able to reach the surface of the char
 particle. So heterogenous combustion can take place : diffusion of
 oxygen from bulk, heterogeneous reaction ({\small kinetically
 limited}) and back diffusion of carbone monoxide. The heterogeneous
 oxidation interfacial mass flux is the difference of an incoming
 oxygen flux and an outcoming carbon monoxide mass flux, each of them
 at their own temperature. The incoming oxygen has a zero valued
 formation enthalpy ({\small reference state}) and the outcoming
 carbon monoxide has a negative formation enthalpy, as a result the
 enthalpy liberated by the first oxidation of carbon is leaked in the
 particle energy, contributiong to its heating. The heterogenous
 combustion is achieved if all the carbon of the char particle is
 converted, leaving an ash particle. Unburnt carbon can leave the
 boiler as flying particle. The heterogeneous reaction is wrotten
 : \\ \centerline{$Char + \frac{1}{2} 0_{2} =(k0het, T0het)=> CO$}
\end{enumerate}

The $2006$ version is able to deal with many class of particles
\fort{NCLA}, each class beeing described by an initial diameter and a
constituting char. Every char, among \fort{NCHAR} is described by a
complete set of parameters : immediate and ultimate analysis, low
heating value ({\small at user choice on raw, dry or pure}) and
kinetic parameters ({\em for both the two competitive pyrolysis
reaction and for heterogeneous reaction}). This allows to describe the
combustion of a mixture or coals or of coal and every material
describable by such evolution kinetics ({\small woods chips ...}). It
is, obviously, possible to mix fuels with ({\small very}) different
proximate analysis, like dry hard coal and wet biomass.


\newpage
%=================================
\subsection{Notations}
%=================================

\begin{table}[h!]
\begin{tabular}{ccp{10,5cm}}

{\bf Symbol} & {\bf Unit} & {\bf Meaning}\\


$H$ 		& $J/kg$ 	& specific enthalpy \\
$K$ 		& $kg/(m.\,s)$ 	& thermal diffusivity\\
$\lambda$ 	& $W/(m.\,K)$ 	& thermal conductivity\\
$\mu$	 	& $kg/(m.\,s)$ 	& dynamical viscosity\\
$\rho$ 		& $kg/m^3$ 	& density\\
$M$, $M_i$ 	& $kg/mol$ 	& molar mass ($M_i$ for  $i$� constituant)\\
$P$ 		& $Pa$ 		& pressure\\
$R$ 		& $J/(mol.\,K)$ & perfect gas constant\\
$T$ 		& $K$ 		& temperature ($>0$)\\
$Y_i$ 		& 		& mass fraction of constituant $i$ 
					($0 \leqslant Y_i \leqslant 1$)\\
$D^{t}$         & $kg/(m.\,s)$  & turbulent viscosity \\
$\alpha_{i}$    & 		& mass fraction of phase k \\
$t$ 		& $s$ 		& time\\
\end{tabular}
\end{table}

\clearpage

%=================================
\subsection{Budget Equations}
%=================================

The bulk, done of gases and particles, is assumed to be describable
with only one pressure and velocity. The slipping velocitiy between
particles and gases is supposed negligible compared to this mean
velocity.  Scalars for the bulk are :
\vspace{0.5cm}
\begin{itemize}
  \item Bulk density 
     \begin{equation} 
        \rho_{m} = \alpha_{1}\rho_{1} + \alpha_{2}\rho_{2}
     \end{equation} 
  \item Bulk velocity 
     \begin{equation} 
       U_{m} = \frac{ \alpha_{1}\rho_{1} U_{1} 
                    + \alpha_{2}\rho_{2} U_{2} }{\rho_{m}}
     \end{equation} 
  \item Bulk enthalpy 
     \begin{equation} 
        H_{m} = \frac{ \alpha_{1}\rho_{1} H_{1} 
                     + \alpha_{2}\rho_{2} H_{2} }{\rho_{m}}
     \end{equation} 
  \item Bulk pressure 
     \begin{equation} 
       P_{m} = P_{1}
     \end{equation} 
\end{itemize}  

Mass fractions of gaseous medium ($Y_{1}^{*}$) and of partciles are defined by :
\begin{eqnarray}
  Y_{1}^{*} = \frac{\alpha_{1}\rho_{1}}{\rho_{m}} &\\
  Y_{2}^{*} = \frac{\alpha_{2}\rho_{2}}{\rho_{m}} &
\end{eqnarray}

So budget equations for the bulk can be written :

\begin{equation}
  \frac{\partial}{\partial t    } \rho_{m}
 +\frac{\partial}{\partial x_{j}} (\rho_{m}U_{m,j}) = 0 
\end{equation}

\begin{equation}
  \frac{\partial}{\partial t    } (\rho_{m}U_{m,i})
 +\frac{\partial}{\partial x_{j}} (\rho_{m}U_{m,i}U_{m,j})
       =  \frac{\partial}{\partial x_{j}}
              \left[ \rho_{m} \left[ D_{m}^{t}( \frac{\partial U_{m,i}}{\partial x_{j}}
                                  +\frac{\partial U_{m,i}}{\partial x_{j}} ) \\
                      -\frac{2}{3}\delta_{ij}
                            ( q_{m}^{2}
                             +D_{m}^{t}\frac{\partial U_{m,l}}{\partial x_{l}}) \right]  \right]
           - \frac{\partial P_{m}}{\partial x_{i}}+\rho_{m}g_{i}
\end{equation}

\begin{equation}
  \frac{\partial}{\partial t    } (\rho_{m} H_{m})
 +\frac{\partial}{\partial x_{j}} (\rho_{m}U_{m,j}H_{m})
              = \frac{\partial}{\partial x_{j}} 
                       (\rho_{m}D_{m}^{t} \frac{\partial H_{m}}{\partial x_{j}})
               +S_{m,R}
\end{equation}
 
With the ({\small velocity}) homogeneity assumption, mainly budget
equation for bulk caracteristic are pertinent. So transport equation
for the scalar $\Phi_{k}$, where k is the phase, can be written :

\begin{equation}
  \frac{\partial}{\partial t    } (\rho_{m} Y_{k}^{*}\Phi_{k})
 +\frac{\partial}{\partial x_{j}} (\rho_{m} U_{m,j} Y_{k}^{*} \Phi_{k})
              = \frac{\partial}{\partial x_{j}} 
                       (\rho_{m}D_{m}^{t} \frac{\partial Y_{k}^{*} \Phi_{k}}{\partial x_{j}})
               +S_{\Phi_{k}}+\Gamma_{\Phi_{k}}
\end{equation}

%=================================
\subsection{Coal combustion scalars}
%=================================

\subsubsection{Bulk enthalpy : $H_{m}$ }
Budget equation for the specific enthalpy of the mixture ({\small gas
+ particles}) admits only one source term for radiative effects
$S_{m,R}$ :
\begin{equation}
    S_{m,R}= S_{1,R}+ S_{2,R}
\end{equation}
With contributions of each phases liable to be described by different
models ({\small eg : wide band for gases, black body for particles}).

\subsubsection{Particles enthalpy : $Y_{2}^{*}H_{2}$ }
Enthalpy of droplets ({\small J in particles/kg bulk}) is the product
of solid phase mass fraction ({\small kg liq/kg bulk}) by the specific
enthalpy of solid ({\small kg solid/kg bulk}). So the budget equation
for liquid enthalpy has six source terms : :
\begin{equation}
     \Pi_{2}^{'}+S_{2,R}-\Gamma_{evap}H_{H2Ovap}(T_{2})-\Gamma_{devol1}H_{MV1}(T_{2})-\Gamma_{devol2}H_{MV2}(T_{2})
                        +\Gamma_{het}\left( \frac{M_{O}}{M_{C}}H_{O_{2}}(T_{1})
                                      -\frac{M_{CO   }}{M_{C}}H_{CO   }(T_{2})\right) 
\end{equation}
with
\begin{itemize}
  \item $\Pi_{2}^{'}$ : heat flux between phases
  \item $S_{2,R}$ : radiative source term for droplets
  \item $\Gamma_{evap}H_{H2Ovap}(T_{2})$ the vapor flux leaves at particle temperature ($H_{vap}$ includes latent heat)
\item $\Gamma_{dvol1}H_{MV1}(T_{2})$ the light volatile matter flux leaves at particle temperature ($H_{vap}$ includes latent heat)
\item $\Gamma_{dvol2}H_{MV2}(T_{2})$ the heavy volatile matter flux leaves at particle temperature ($H_{vap}$ includes latent heat)
   \item $\Gamma_{het}(...)$ heterogenous combustion induces reciprocal mass flux : oxygen arriving at gas temperature and carbone monoxide leaving at char particle one.

\end{itemize}

\subsubsection{Dispersed phase mass fraction : $Y_{2}^{*}$}
In budget equation for the mass fraction of the dispersed phase
({\small first droplets, then char particles, at last ashes}) the
source terms are interfacial mass fluxes ({\small first evaporation,
then net flux for heterogeneous combustion}):

\begin{equation}
     -\Gamma_{evap}-\Gamma_{het}
\end{equation}
          
\subsubsection{Number of particles : $N_{p}^{*}$}
No source term in the budget equation for number of droplets : a
droplet became a particle ({\small eventually a tiny flying ash}) but
never vanish ({\small particles it have to get out}).

                                
\subsubsection{Mean of the passive scalar for light volatile : $F_{1}$}  
This scalar represent the amount of matter which have leaved the
particle as fuel vapour, whatever it happens after. It's a mass
fraction of gaseous matter ({\small in hydrocarbon form or carbon
oxide ones}). So the source term in its budget is only evaporation
mass flux :
\begin{equation}
   \Gamma_{evap}
\end{equation}     

\subsubsection{ Variance of $F_{1}$ : $F_{1}^{'2}$}
Budget equation for $F_{1}^{'2}$ have three source term :
\begin{equation}
   \Gamma_{F_{1}^{'2}}
   -2\rho_{m}Y_{1}^{*}\frac{F_{1}^{'2}}{\tau_{\chi_{F_{1}^{'2}}}} 
   + \rho_{m}Y_{1}^{*}D_{m}^{t}\frac{\partial F_{1}}{\partial x_{j}} 
                               \frac{\partial F_{1}}{\partial x_{j}}
\end{equation} 
where $\Gamma_{F_{1}^{'2}}$ is due to interfacial mass fluxes ({\small
every interfacial mass fluxes impact gaseous phase variances}).
                                                 
\subsubsection{Mean of the passive scalar for carbon from char : $F_{3}$}  
Budget equation for $F_{3}$ have for lone source term the mass flux
due to heterogeneous combsution ({\small mass flux of carbon monoxide
minus oxygene mass flux}). As for $F_{1}$ oxidation in the gaseous
phase does not modifiy this {\em passive} scalar :
\begin{equation}
   \Gamma_{het}
\end{equation}   
         
                       
\subsubsection{ Variance of the passive scalar for air : $F_{4}^{'2}$} 
 
This passive scalar incomes with air but is'nt destroyed by any
({\small in gaseous phase or heterogeneous}). No budget equation
needed for it, $F_{4}$ can be determined from the wholeness
relation.\\

Budget equation for $F_{4}^{'2}$ have, like other passive scalar
variance budget equaiton, four source terms :
\begin{equation} 
    \Gamma_{F_{4}^{'2}}
   -2\rho_{m}Y_{1}^{*}\frac{F_{4}^{'2}}{\tau_{\chi_{F_{4}^{'2}}}} 
   + \rho_{m}Y_{1}^{*}D_{m}^{t}\frac{\partial F_{4}}{\partial x_{j}} 
                               \frac{\partial F_{4}}{\partial x_{j}}
\end{equation} 
where $\Gamma_{F_{4}^{'2}}$ is due to interfacial mass fluxes ({\small every interfacial mass fluxes impact gaseous phase variances}).



%%%%%%%%%%%%%%%%%%%%%%%%%%%%%%%%%%
%%%%%%%%%%%%%%%%%%%%%%%%%%%%%%%%%%
\section{Discr\'etisation}
%%%%%%%%%%%%%%%%%%%%%%%%%%%%%%%%%%
%%%%%%%%%%%%%%%%%%%%%%%%%%%%%%%%%%

On se reportera aux sections relatives aux sous-programmes
\fort{cfmsvl} (masse volumique), \fort{cfqdmv} 
(quantit\'e de mouvement) et \fort{cfener} (\'energie).  La
documentation du sous-programme
\fort{cfxtcl} fournit des \'el\'ements relatifs aux 
conditions
aux limites. 
