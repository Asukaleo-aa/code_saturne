%CVERS    CODE_SATURNE VERSION 1.3              COPYRIGHT EDF 1999-2006 *
%---------------------------------------------------------------------
%
% Auteur  :
% Date    :
% Num�ro  :
% Version :
%
%---------------------------------------------------------------------

%%%%%%%%%%%%%%%%%%%%%%%%%%%%%%%%%%%%%%%%%%%%%%%%%%%%%%%%%%%%%%%%%%%%%%
%                                                                    %
%                                                                    %
%                                                                    %
% Titre :           Manuel Utilisateur de Code_Saturne               %
%                                                                    %
%                                                                    %
%                                                                    %
%%%%%%%%%%%%%%%%%%%%%%%%%%%%%%%%%%%%%%%%%%%%%%%%%%%%%%%%%%%%%%%%%%%%%%
\documentclass[a4paper,10pt,twoside]{article}

%
%%%%%%%%%%%%%%%%%%%%%%%%%%%%%%%%%%%%%%%%%%%%%%%%%%%%%%%%%%%%%%%%%%%%%%
% PACKAGES OBLIGATOIRES
\usepackage[rddoc]{noteEDF}
%
% Les options de la classe noteEDF.sty disponibles sont :
% - english (page de garde supplementaire en anglais)
% - pdftex
% Les 3 options suivantes sont exclusives mutuellement :
% - confidentiel (orientation dans le fonds documentaire et accesibilit�)
% - rddoc (orientation dans le fonds documentaire)
% - edfdoc (orientation dans le fonds documentaire)
%
%%%%%%%%%%%%%%%%%%%%%%%%%%%%%%%%%%%%%%%%%%%%%%%%%%%%%%%%%%%%%%%%%%%%%%

%
%%%%%%%%%%%%%%%%%%%%%%%%%%%%%%%%%%%%%%%%%%%%%%%%%%%%%%%%%%%%%%%%%%%%%%
% PACKAGES ET COMMANDES POUR LE DOCUMENTS PDF ET LES HYPERLIENS


%
%%%%%%%%%%%%%%%%%%%%%%%%%%%%%%%%%%%%%%%%%%%%%%%%%%%%%%%%%%%%%%%%%%%%%%
% PACKAGES SUPPLEMENTAIRES
% \usepackage[...]{.....}
%
% Packages charg�s par noteEDF.sty :
% graphicx, tabularx, amsmath, amssymb, wasysym,
% fancyhdr, lastpage, ifthen, setspace,
% [latin1]{inputenc}, [OT1]{fontenc}, [francais]{babel}
%
%\usepackage{times} % plus joli en pdf (fortement recommand�)
%\usepackage{here} % option [H] pour que les objets flottant ne flottent plus
\usepackage{xspace}
%
%%%%%%%%%%%%%%%%%%%%%%%%%%%%%%%%%%%%%%%%%%%%%%%%%%%%%%%%%%%%%%%%%%%%%%

%
%%%%%%%%%%%%%%%%%%%%%%%%%%%%%%%%%%%%%%%%%%%%%%%%%%%%%%%%%%%%%%%%%%%%%%
% MACROS SUPPLEMENTAIRES
% \newcommand{/...}{...}
%
% repertoire et extension des images
\newcommand{\repgraphics}{../graphics}
\newcommand{\extgraphics}{pdf}
%
\newcommand{\verscs}{1.3.6\xspace}
\newcommand{\CS}{%
{\fontfamily{ppl}\fontshape{it}\selectfont Code\_Saturne}\xspace}
\newcommand{\syrthes}{SYRTHES\xspace}
%
\newcommand{\tildeunix}{%
{\huge$_{_{\widetilde{\ }}}$}\hspace*{0.1mm}}
%
%(Attention : \passepage ne fonctionne pas devant un \chapter (classe 'report'))
\newcommand{\passepage}{%
\ifthenelse{\isodd{\arabic{page}}}
{\newpage\hspace*{6.cm}\newpage}{\newpage}}
%
\newcommand{\minititre}[1]{\bigskip\noindent \underline{\sc #1}\\}
\newcommand{\vect}[1]{\underline{#1}}
\newcommand{\tens}[1]{\underline{\underline{#1}}}
\newcommand{\grad}{\text{g}\underline{\text{rad}}\ }
\newcommand{\ggrad}{\text{g}\underline{\underline{\text{rad}}}\ }
\newcommand{\dive}{\text{div}}
\newcommand{\ind}[1]{\text{$_{#1}$}}
\newcommand{\degresC}{$^\circ$C}
\newbox\tempbox
\newcommand{\motcle}[7]{%
   \noindent
   \setbox\tempbox\hbox{\hspace*{2.5cm}}
   \makebox[2.5cm][l]{#1}\index{#1}\makebox[1.3cm][l]{#2}\makebox[6.cm][l]{#3}%
   \makebox[4.cm][l]{[#4]}#5\hspace{1cm}#6\\
   \hangindent\wd\tempbox\quad\ignorespaces#7\bigskip}
\newcommand{\motcleb}[7]{%
   \noindent
   \setbox\tempbox\hbox{\hspace*{2.5cm}}
   \makebox[2.5cm][l]{\bf #1}\index{#1}\makebox[1.3cm][l]{#2}\makebox[6.cm][l]{#3}%
   \makebox[4.cm][l]{[#4]}#5\hspace{1cm}#6\\
   \hangindent\wd\tempbox\quad\ignorespaces#7\bigskip}
%
%\newcommand{\variab}[4]{%
%       \hangindent\wd\tempbox\noindent{#2\index{#1} [#3] \quad\\}
%       \hspace*{0.5cm}\ignorespaces#4.}
\newcommand{\variab}[4]{%
       \setbox\tempbox\hbox{IFACEL [E] : }
       \hangindent\wd\tempbox\noindent{#2\index{#1} [#3] :\quad}\ignorespaces#4.}
\newcommand{\variablist}[4]{%
       \setbox\tempbox\hbox{IFACEL [E] : }
       \hangindent\wd\tempbox\noindent{#2\index{#1} [#3] :\quad}\ignorespaces#4\quad}
%\newcommand{\variab}[4]{%
%       \setbox\tempbox\hbox{#2 [#3] :\quad}
%       \hangindent\wd\tempbox\noindent{#2\index{#1} [#3] :\quad}\ignorespaces#4.}
%
%%%%%%%%%%%%%%%%%%%%%%%%%%%%%%%%%%%%%%%%%%%%%%%%%%%%%%%%%%%%%%%%%%%%%%

%
%%%%%%%%%%%%%%%%%%%%%%%%%%%%%%%%%%%%%%%%%%%%%%%%%%%%%%%%%%%%%%%%%%%%%%
% INFO POUR PAGES DE GARDES
\titreEDFfr{Guide pratique d'utilisation de \CS version~\verscs}
\titreEDFang{\CS version~\verscs\ practical user's guide}
\numeroEDF{HI-83/--/---}
\indiceEDF{A}

\authorEDFfr{\'Equipe de d�veloppement de \CS}
\authorEDFang{\CS development team}
%\rqauteursEDF{* Sigle de la structure du coauteur EDF, si diff�rente de l'auteur principal.}
%\remauteursEDF{** Nom de la soci�t� (en clair) si le coauteur est ext�rieur.}
\rqauteursEDF{}
\remauteursEDF{}

%\groupeEDFfr{Expertise en bidules et trucs}
%\groupeEDFang{Very strong in something}

\docassociesEDFfr{}
\docassociesEDFang{}
\resumeEDFfr{}
\resumeEDFang{This document presents all the necessary elements to run a calculation
with \CS version \verscs. It then lists all the variables of the code
which may be useful for more advanced utilisation.
The user subroutines of all the modules within the code are also documented.
Eventually, for each key word and user-modifiable parameter in the code,
their definition, allowed values, default values and conditions for use are given.
These key words and parameters are grouped under headings
based on their function. An alphabetical index list is also given at the end of
the document for easier consultation.}
\accessibiliteEDFfr{RESTREINTE} % LIBRE, EDF-GDF ou RESTREINTE
\accessibiliteEDFang{RESTRICTED} % FREE, EDF-GDF or RESTRICTED

\actionEDF{Projet JUPITER}
\classementEDF{}
\typerapportEDF{Note technique} % Note d'�tude, Note technique
\motsclesEDF{\CS, CFD, thermohydraulique}
%
%%%%%%%%%%%%%%%%%%%%%%%%%%%%%%%%%%%%%%%%%%%%%%%%%%%%%%%%%%%%%%%%%%%%%%

%
%%%%%%%%%%%%%%%%%%%%%%%%%%%%%%%%%%%%%%%%%%%%%%%%%%%%%%%%%%%%%%%%%%%%%%
% DEBUT DU DOCUMENT

\begin{document}
%%%%%%%%%%%%%%%%%%%%%%%%%%%%%%%%%%%%%%%%%%%%%%%%%%%%%%%%%%%%%%%%%%%%%%%%%%%%%%%%%%%%%%%%%%%%%%%%%%%%%%%%%%%%%%%%
%==================================
\subsection{Ale module}
%==================================
The ale module can be used in two ways :
\begin{list}{-}{}
  \item pure ale
  \item ale with a structure module
\end{list}


%==================================
\subsubsection{ Initialization of the options of the variables related
   to the ale module: \textmd{\texttt{usalin}} and \textmd{\texttt{usstr1}} }
%==================================
\label{prg_usalin}%
\noindent
\textit{Subroutine called at the start.}


\minititre{Subroutine \texttt{usalin}}
This subroutine completes \texttt{usini1}.

\texttt{usalin} allows to set option for the ale module, and in particular to active the ale module

\minititre{Subroutine \texttt{usstr1}}
\texttt{usstr1} allows to specify for the structure module the following pieces of information:
\begin{list}{-}{}
  \item number of structure (NBSTRU).
  \item initial value of deplacement, velocity and acceleration (XSTR0, XSTREQ and VSTR0).
\end{list}

Below is a list of the different variables that might be modified :

\begin{list}{$\bullet$}{}

\item{NBSTRU} \\
{the number of structures}

\item{IDFSTR(I)} \\
{index of the structure, where I is the index of the face}

\item{XSTR0(I,K)} \\
{ initial position of a structure, where I is the dimension of space and K the index of the structure
}

\item{XSTREQ(I,K)} \\
{position of balance of a structure, where I is the dimension of space and K the index of the structure
}

\item{VSTR0(I,K)} \\
{initial velicity of a structure, where I is the dimension of space and K the index of the structure
}
\end{list}

%==================================
%\subsubsection[Gestion des conditions aux limites de vitesse de maillage ale compressible : \texttt{usalcl}]
\subsubsection{Management of the boundary conditions of velocity mesh related to the
   ale module: \textmd{\texttt{usalcl}}}
%==================================


\noindent
\textit{Subroutine called every time step.}

\minititre{Subroutine \texttt{usalcl}}
The use of \texttt{usalcl}
is obligatory to run a calculation using the ale module just
as it is in \texttt{usini1}  . The way of using it is the same as the
way of using \texttt{usclim} in the framework of standard calculations, that is to
say a loop on the boundary faces marked out by their colour (or more
generally by a property of their family), where the type of boundary
condition of velocity mesh for each variable are defined.

The main numerical variables are described below.

\variab{IALTYB}{IALTYB(NFABOR)}{IA}{In the ale module, the user
defines the velocity mesh from the colour of the
boundary faces, or more generally from their properties (colours, groups
...), from the boundary conditions defined in \texttt{usclim}, or even
from their coordinates. To do so, the array IALTYB(NFABOR) gives for
each face IFAC the velocity mesh boundary condition types
marked out by the key words IVIMPO\index{IVIMPO}, IGLISS\index{IGLISS}, IBFIXE\index{IBFIXE}

\begin{list}{$\bullet$}{}

\item If IALTYB=IVIMPO : imposed velocity.

\begin{list}{$\rightarrow$}{}
\item In the case where all the nodes of a face have a imposed displacement,
it is not necessary to fill the tables with boundary conditions velocity mesh for this face,
they will be erased. In the other case, the value of the Dirichlet must be given in
RCODCL(IFAC,IVAR,1) for every value of IVAR (IUMA, IVMA and IWMA)
The other boxes of RCODCL and ICODCL are completed automatically.

 The tangential velocity mesh is taken like a tape speed under the boundary conditions
 of wall for the fluid, except if wall velocity was specified by the user in the interface or usclim
 (in which case it is this speed which is considered).
\end{list}

 \item if IALTYB(NFAC) = IBFIXE : fixe wall
\begin{list}{$\rightarrow$}{}
 \item the velocity is null.
\end{list}

 \item if IALTYB(NFAC) = IGLISS :  slice wall
\begin{list}{$\rightarrow$}{}
\item the tangential velocity is not used.
\end{list}

\end{list}

}




\subsubsection{Management of the structure property: \textmd{\texttt{usstr2}}}
%==================================


\noindent
\textit{Subroutine called every time step.}

The use of \texttt{usstr2}
is obligatory to run a calculation using the ale module with a structure module.

For each structure, the system that will be solved is :

 \begin{equation}
M.x^{''}+C.x^{''}+K.(x-x_{0} = 0
\end{equation}

where

\begin{list}{-}{}
 \item $M$ is the mass stucture (XMSTRU).
 \item $C$ is the dumping coefficient of the stucture (XCSTRU).
 \item $K$ is the spring constant or force constant of the stucture (XKSTRU).
 \item $x_{0}$ is the initial position
\end{list}

Below is a list of the different variables that might be modified :

\begin{list}{$\bullet$}{}

\item{XMSTRU(I,J,K)} \\
{the mass stucture of the structure, where I,J is the array of mass structure
and K the index of the structure.
}

\item{XCSTRU(I,J,K)}\\
{dumping coefficient of the stucture, where I,J is the array of dumping coefficient
and K the index of the structure.
}

\item{XKSTRU(I,J,K)}\\
{spring constant of the stucture, where I,J is the array of spring constant
and K the index of the structure.
}

\item{FORSTR(I,K)}\\
{force vector of the stucture, where I is the force vector and K the index of the structure.
}
\end{list}




%%%%%%%%%%%%%%%%%%%%%%%%%%%%%%%%%%%%%%%%%%%%%%%%%%%%%%%%%%%%%%%%%%%%%%%%%%%%%%%%%%%%%%%%

\minititre{ALE}

\motcle{IALE}{I}{0 or 1}{C}{O}{L1}
{activates (=1) or not (=0), activate the ALE module}

\motcle{NALINF}{I}{0 or positive integer}{C}{0}{L2}
{The number of under-iterations of initialization of the fluid}

\motcle{NBSTR}{I}{0 or positive integer}{C}{0}{L1}
{number of structure}

\motcle{ALPNMK}{R}{real}{C}{0.D0}{L3}
{$alpha$ newmark's method}

\motcle{BETNMK}{R}{real}{C}{-GRAND}{L3}
{$beta$ newmark's method}

\motcle{GAMNMK}{R}{real}{C}{-GRAND}{L3}
{$gamma$ newmark's method}

\motcle{NALIMX}{I}{positive integer}{C}{15}{L2}
{maximum number of iterations of implicitation of the structure displacement }

\motcle{EPALIM}{R}{positive real}{C}{1.D-5}{L2}
{Relative precision of implicitation of the structure displacement }



\end{document}
