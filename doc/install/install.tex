%-----------------------------------------------------------------------
%
%     This file is part of the Code_Saturne Kernel, element of the
%     Code_Saturne CFD tool.
%
%     Copyright (C) 1998-2011 EDF S.A., France
%
%     contact: saturne-support@edf.fr
%
%     The Code_Saturne Kernel is free software; you can redistribute it
%     and/or modify it under the terms of the GNU General Public License
%     as published by the Free Software Foundation; either version 2 of
%     the License, or (at your option) any later version.
%
%     The Code_Saturne Kernel is distributed in the hope that it will be
%     useful, but WITHOUT ANY WARRANTY; without even the implied warranty
%     of MERCHANTABILITY or FITNESS FOR A PARTICULAR PURPOSE.  See the
%     GNU General Public License for more details.
%
%     You should have received a copy of the GNU General Public License
%     along with the Code_Saturne Kernel; if not, write to the
%     Free Software Foundation, Inc.,
%     51 Franklin St, Fifth Floor,
%     Boston, MA  02110-1301  USA
%
%-----------------------------------------------------------------------
\documentclass[a4paper,10pt,twoside]{article}

%
%%%%%%%%%%%%%%%%%%%%%%%%%%%%%%%%%%%%%%%%%%%%%%%%%%%%%%%%%%%%%%%%%%%%%%
% PACKAGES OBLIGATOIRES
\usepackage{csdoc}
% MACROS SUPPLEMENTAIRES
\usepackage{csmacros}
\usepackage[usenames, dvipsnames]{color}
%
%%%%%%%%%%%%%%%%%%%%%%%%%%%%%%%%%%%%%%%%%%%%%%%%%%%%%%%%%%%%%%%%%%%%%%

%
%%%%%%%%%%%%%%%%%%%%%%%%%%%%%%%%%%%%%%%%%%%%%%%%%%%%%%%%%%%%%%%%%%%%%%
% PACKAGES ET COMMANDES POUR LE DOCUMENTS PDF ET LES HYPERLIENS
\hypersetup{%
  pdftitle = {CodeSaturne installation guide},
  pdfauthor = {MFEE},
  pdfpagemode = UseOutlines
}
\pdfinfo{/CreationDate (D:20100802000000-01 00 )}
%
% To have thumbnails upon opening the document under ACROREAD
% pdfpagemode = UseThumbs
%
%%%%%%%%%%%%%%%%%%%%%%%%%%%%%%%%%%%%%%%%%%%%%%%%%%%%%%%%%%%%%%%%%%%%%%
%
%%%%%%%%%%%%%%%%%%%%%%%%%%%%%%%%%%%%%%%%%%%%%%%%%%%%%%%%%%%%%%%%%%%%%%
% Info for title pages
\titreCS{\CS version~\verscs installation guide}

\docassociesCS{}
\resumeCS{This document presents all the necessary elements to install
with \CS version \verscs.

\begin{center}
\large{WORK IN PROGRESS}
\end{center}
}
%
%%%%%%%%%%%%%%%%%%%%%%%%%%%%%%%%%%%%%%%%%%%%%%%%%%%%%%%%%%%%%%%%%%%%%%
% Document start
\begin{document}

\def\contentsname{\textbf{\normalsize TABLE OF CONTENTS}\pdfbookmark[1]{Table of
contents}{contents}}

\pdfbookmark[1]{Flyleaf}{pdg}
\large
\makepdgCS
\normalsize

\passepage

\begin{center}\begin{singlespace}
\tableofcontents
\end{singlespace}\end{center}

\section{\CS subsystems\label{sec:subsystems}}

The \CS CFD tool is built of several libraries, some of which are optional,
and which may themselves use third-party tools.
Base libraries on which others are built must be installed first,
The libraries comprising \CS are the following (in recommended
order of installation).

\begin{itemize}
\item BFT (Base Functions and Types) is mainly a portability layer
      for the rest of the code. It is written in ANSI C, and is used by
      all the other parts of the code, so it should be installed first.
\item the \pcs (originally named ``Enveloppe \CS'', or ECS) also comprises
      the optional partitioner, and a preprocessed file dump tool.
      Building \CS without this package is possible, if it is available
      on another machine, but it is a part of any standard build.
      In addition to the required BFT library, this package may use
      third-party libraries such as MED, CGNS, libCCMIO to allow import of
      the corresponding mesh formats, and \metis, and \scotch for
      domain partitioning.
\item FVM (Finite Volume Mesh) is used to provide finite volume mesh
      I/O and some other services such as interpolation, possibly in
      parallel using MPI. This package requires the BFT library,
      and may use third-party libraries such as MED and CGNS
      to allow export to the corresponding mesh formats. To build a
      \CS with parallel execution support, FVM must be built with MPI.
\item MEI (Mathematical Expression Interpreter) is used to allow
      definition of complex properties and boundary conditions using
      mathematical expressions defined through the GUI, avoiding
      the need for user subroutines in many cases. This package
      is optional, as it is only useful when the GUI is used.
      It requires the BFT library.
\item The \CS solver, or kernel (NCS stands for ``Noyau \CS'', or
      \CS Kernel in French) is the major and (final) part of the
      installation. It requires BFT, FVM, and optionally MEI.
      For parallel execution, it must be built with the same MPI library
      as FVM.
      This module also contains the optional GUI and most of the code's
      documentation.
\end{itemize}

\section{Installation basics\label{sec:install_basics}}

The installation scripts of \CS are based on the GNU Autotools,
(Autoconf, Automake, and Libtool), so it should be familiar for many
administrators. A few remarks are given here:

\begin{itemize}
\item As with most software with modern build systems, it is recommended
      to build the code in a separate directory from the sources. This
      allows multiple builds (for example production and debug), and is
      considered good practice. Building directly in the source tree is
      not regularly tested, and is not guaranteed to work, in addition
      to ``polluting'' the source directory with build files.
\item By default, optional libraries which may be used by \CS are
      enabled automatically if detected in default search paths
      (i.e. \texttt{/usr/} and \texttt{/usr/local}. To find libraries
      associated with a package installed in an alternate path,
      a \texttt{--with-<package>=...} option to the \texttt{configure} script
      must given. To disable the use of a library which would be
      detected automatically, a matching \texttt{--without-<package>} option
      must be passed to \texttt{configure} instead.
\item Most third-party libraries usable by \CS are considered optional,
      and are simply not used if not detected, but the libraries needed by
      the GUI are considered mandatory, unless the \texttt{--disable-gui}
      or \texttt{--disable-frontend} option is explicitly used.
\end{itemize}

The following chapters give more details on \CS's recommended
third-party libraries, configuration recommendations, troubleshooting,
and post-installation options.

\section{Third-Party libraries\label{sec:ext_lib}}

For a minimal build of \CS, a Posix system with a C and a Fortran compiler,
a Python interpreter and a {\tt make} tool should be sufficient.
For parallel runs, an MPI library is also necessary.
To build an use the GUI, Libxml2 and PyQt4 (which in turn requires
Qt4 and SIP) are required.
Other libraries may be used for additional mesh format options,
as well as to improve performance. A list of those libraries
and their role is given in \S\ref{sec:list_ext_lib}.

\subsection{Installing third-party libraries for \CS\label%
{sec:obtain_ext_lib}}

Third-Party libraries usable with \CS may be installed in several
ways:

\begin{itemize}
\item On many Linux systems, most of libraries listed in
      \S\ref{sec:list_ext_lib} are available through the distribution's
      package manager.\footnote{On Mac OS X systems, package managers such as
      Fink or MacPorts also provide package management, even though the
      base system does not.} This requires administrator privileges,
      but is by far the easiest way to install third-party libraries
      for \CS.

      Note that distributions usually split libraries or tools into runtime
      and development packages, and that although some packages are
      installed by default on many systems, this is generally not the
      case for the associated development headers. Development
      packages usually have the same name as the matching runtime package,
      with a \texttt{-dev} postfix added. For example, on a Debian or
      Ubuntu system ,\texttt{libxml2} is usually installed by default,
      but \texttt{libxml2-dev} must also be installed for the \CS
      build to be able to use the former.

\item On many large compute clusters, Environment Modules allow
      the administrators to provide multiple versions of many
      scientific libraries, as well us compilers or MPI libraries,
      using the \texttt{module} command. More details on
      Environment Modules may be found at \url{http://modules.sourceforge.net}.
      When being configured and installed \CS checks for modules loaded
      with the \texttt{module} command, and records the list of loaded
      modules. Whenever running that build of \CS, the modules detected
      at installation time will be used, rather than those defined by
      default in the user's environment. This allows using versions of
      \CS built with different modules safely and easily, even if the
      user may be experimenting with other modules for various purposes.

\item If not otherwise available, third-party software may be compiled
      an installed by an administrator or a user. An administrator
      will choose where software may be installed, but for a user
      without administrator privileges or write access to
      \texttt{usr/local}, installation to a user account is often
      the only option. None of the third-party libraries usable
      by \CS require administrator privileges, so they may all be
      installed normally in a user account, provided the user
      has sufficient expertise to install them. This is usually not
      complicated (provided one reads the installation instructions,
      and is prepared to read error messages if something goes wrong),
      but even for an experienced user or administrator, compiling
      and installing 5 or 6 libraries as a prerequisite significantly
      increases the effort required to install \CS.

      Even though it is more time-consuming, compiling and installing
      third-party software may be necessary when no matching packages
      or Environment Modules are available, or when a more recent version
      or a build with different options is desired.
\end{itemize}

\subsection{List of third-party libraries usable by \CS\label%
{sec:list_ext_lib}}

The list of third-party software usable with \CS is provided here:

\begin{itemize}

\item BLAS (Basic Linear Algebra Subroutines) may be used by the
      Kernel (NCS) for certain operations such as vector sums, dot
      products, etc. If no third-party BLAS is provided, \CS reverts to
      its own implementation of BLAS routines, similar to the legacy
      Netlib BLAS, so no functionality is lost here. With optimized BLAS
      such as Atlas, MKL, ESSL, and others, simple operations such as vector
      dot product, $Ax+y$, and such are often an order of magnitude
      faster than with the usual compiler optimizations, leading to an
      overall performance increase of about 10\%.

\item PyQt4 is required by the \CS GUI (part of NCS). PyQt4 in turn
      requires Qt4, Python,and SIP. Without this library, the GUI may not
      be built, although XML files generated with another install of \CS
      may be used if Libxml2 is available.

\item SWIG is required to build MEI Python bindings for the GUI
      (Python development headers are also necessary). Without this
      tool, the GUI will not allow the user to define MEI expressions,
      although the Solver will still be able to interpret such expressions
      when built with MEI and using an XML file produced by an install
      using MEI and its Python bindings. A minimum of SWIG version
      1.3.30 is required, due to bugs in older versions.

\item Libxml2 is required by the Kernel (NCS) to read XML files edited
      with the GUI. If this library is not available, only user subroutines
      may be used to setup data.

\item HDF5 is necessary for MED, and may also be used by CGNS. It may be
      used by the \pcs for mesh import, FVM for post-processing output.

\item CGNSlib is necessary to read or write mesh and visualization files
      using the CGNS format, available as an export format with many
      third-party meshing tools. It may be used by the \pcs for
      mesh import, FVM for post-processing output (the former is usually
      more useful than the latter, as the built-in EnSight output is
      readable by most post-processing tools).

\item MED is necessary to read or write mesh and visualization files
      using the MED format, mainly used by the SALOME platform.
      It may be used by the \pcs for mesh import, FVM for
      post-processing output (the former is usually more useful than
      the latter, as the built-in EnSight output is readable by most
      post-processing tools).

\item libCCMIO may be used by the \pcs to import mesh files generated by
      \starccmp using its native format.

\item \metis  may be used by the partitioner, part of the ECS (\pcs)
      package, to optimize mesh partitioning.
      Depending on the mesh, parallel computations with meshes partitioned
      with this library may be from 10\% to 50\% faster than using the
      built-in space-filling curve based partitioning.

      Though broadly available, the license is quite restrictive,
      so \scotch may be preferred.

\item \scotch is an alternative mesh partitioning library, and may also
      be used by the ECS (\pcs) package.
      quality is usually slightly higher than that obtained with \metis,
      though this library is not as fast.

\end{itemize}

For developers, the GNU Autotools (Autoconf, Automake, Libtool) as
well as gettext will be necessary. To build the documentation,
pdf\LaTeX{} and \texttt{fig2dev} (part of TransFig) will be necessary.

\subsection{Notes on some third-party tools and libraries}

\subsubsection{Python and PyQt4\label{sec:ext:python}}

\CS requires a Python interpreter, with Python version 2.4 or above.
The base scripts should work both with Python 2 or Python 3 versions,
but have not been tested recently with the latter. The GUI is Python 2
only, so using Python 3 is not currently recommended.

While \CS makes heavy use of Python, this is for scripts and for the GUI only;
The solver only uses compiled code, so we may for example use
a 32-bit version of Python with 64-bit \CS libraries and executables.

The GUI is written in PyQt4 (Python bindings for Qt4), so but Qt4 and
the matching Python bindings must be available. On most modern
Linux distributions, this is available through the package manager,
which is by far the preferred solution. When running on a system which does
not provide these libraries, there are several alternatives:

\begin{itemize}

\item build \CS without the GUI. If built with Libxml2, XML files
      produced with the GUI are still usable, so if an install of \CS
      with the GUI is available on an other machine, the XML files
      may be copied on the current machine. This is certainly not an optimal
      solution, but in the case where users have a mix of desktop or virtual
      machines with modern Linux distributions and PyQt4 installed, and
      a compute cluster with an older system, this may avoid requiring
      a build of Qt4 and PyQt4 on the cluster if users find this too daunting.

\item Install a local Python interpreter, and add Qt4 bindings to this
      interpreter.

      Python (\url{http://www.python.org}) and Qt4
      (\url{http://qt.nokia.com/products}) must be downloaded  and
      installed first, in any order. The installation instructions of
      both of these tools are quite clear, and though the installation of these
      large packages (especially Qt4) may be a lengthy process in terms of
      compilation time, but is well automated and usually devoid of nasty
      surprises.\footnote{The only case in which the \CS developers
      have has issues with Qt4 was when trying to force an install into
      64-bit mode with the GNU compilers (version 4.1.2) on a PowerPC 64
      architecture running SLES 10 Linux, on which compilers default
      to building 32 bit code, although 64 bit is available. Using default
      options on the same machine led to a perfectly functional 32-bit Qt
      installation}.

      Once Python is installed, the SIP bindings generator
      (\url{http://riverbankcomputing.co.uk/software/sip/intro})
      must also be installed. This is a small package, and configuring it
      simply requires running \texttt{python configure.py} in its source
      tree, using the Python interpreter just installed.

      Finally, the PyQt4 bindings
      (\url{http://riverbankcomputing.co.uk/software/pyqt/intro}) may be
      installed, in a manner similar to SIP.

      When this is finished, the local Python interpreter contains
      the PyQt4 bindings, and may be used by \CS's \texttt{configure}
      script by passing \texttt{PYTHON=<path\_to\_python\_executable}.
     
\item add Python Qt4 bindings as a Python extension module for an existing
      Python installation. This is a more elegant solution than the previous
      one, and avoids requiring rebuilding Python, but if the user does not
      have administrator privileges, the extensions will be placed in a
      directory that is not on the default Python extension search path, and
      that must be added to the \texttt{PYTHONPATH} environment variable.
      This works fine, but for all users using this build of \CS, the
      \texttt{PYTHONPATH} environment variable will need to be
      set.\footnote{In the future, the \CS installation scripts could check
      the \texttt{PYTHONPATH} variable and save its state in the build so as
      to ensure all the requisite directories are searched for.}

      The process is similar to the previous one, but SIP and PyQt4
      installation requires a few additional configuration options
      in this case. See the SIP and PyQt4 reference guides for
      detailed instructions, especially the \emph{Building a Private
      Copy of the SIP Module} section of the SIP guide.

\end{itemize}

\subsubsection{\med\label{sec:ext:med}}

The Autotools installation of MED is simple on most machines,
but a few remarks may be useful for specific cases.

MED has a C API, is written in a mix of C and C++ code,
and provides both a C (\texttt{libmedC}) and an Fortran API
(\texttt{libmed}). Both libraries are always built, so a Fortran
compiler is required, but \CS only links using the C API, so using
a different Fortran compiler to build MED and \CS is possible.

MED does require a C++ runtime library, which is usually transparent
when shared libraries are used. When built with static libraries
only, this is not sufficient, so when testing for a MED library,
the \CS \texttt{configure} script also tries linking with a C++
compiler if linking with a C compiler fails. This must be the
same compiler that was used for MED, to ensure the runtime matches.

Also, when building MED in a cross-compiling situation,
\texttt{--med-int=int} or \texttt{--med-int=int64\_t} (depending
on whether 32 or 64 bit ids should be used) should be
passed to its \texttt{configure} script to avoid a run-time
test.

\section{Preparing for build\label{sec:prepare}}

If the code was obtained as an archive, its components must be unpacked:

\fbox{\begin{minipage}{\textwidth}\texttt{\\
\$ tar xvzf bft-1.1.5.tar.gz\\
\$ tar xvzf ecs-2.0.2.tar.gz\\
\$ tar xvzf fvm-0.15.3.tar.gz\\
\$ tar xvzf mei-1.0.1.tar.gz\\
\$ tar xvzf ncs-2.0.2.tar.gz
}\end{minipage}}

If the packages are provided as zip files, \texttt{unzip} may be substituted
for \texttt{tar xvzf}.
If for example you unpacked the directory in a directory
named \texttt{/home/user/Code\_Saturne/2.0/src}, you will now
have sub-directories named \texttt{bft-1.1.5}, \texttt{ecs-2.0.2},
\texttt{fvm-0.15.3}, \texttt{mei-1.0.1}, and \texttt{ncs-2.0.2}.

For developers obtaining the code was obtained through a version control
system such as Subversion, running a \texttt{./sbin/bootstrap}
script is also required for each package; for example, for BFT:

\fbox{\begin{minipage}{\textwidth}\texttt{\\
\$ cd bft-1.0.5\\
\$ ./sbin/bootstrap\\
\$ cd ..
}\end{minipage}}

This script requires the GNU Autotools (Autoconf, Automake, and Libtool)
to be installed.

It is recommended to build the code in a separate directory from the source.
This also allows multiple builds, for example, building both an
optimized and a debugging version. In this case, choose a consistent
naming scheme, using an additional level of sub-directories,
for example, for BFT:

\fbox{\begin{minipage}{\textwidth}\texttt{\\
\$ mkdir bft-1.0.5\_build\\
\$ cd bft-1.0.5\_build\\
\$ mkdir prod\\
\$ cd prod
}\end{minipage}}

Some older system's {\tt make} command may not support compilation
in a directory different from the source directory ({\tt VPATH}
support). In this case, installing and using the GNU {\tt gmake}
tool instead of the native {\tt make} is recommended. 

\section{Configuration\label{sec:config}}

\CS uses a build system based on the GNU Autotools, which includes
its own documentation.

To obtain the full list of available configuration options for a given
package, run: {\tt configure~--help}.

Note that for all options starting with {\tt --enable-}, 
there is a matching options with {\tt --disable-}. Similarly,
for every {\tt --with-}, {\tt --without-} is also possible.

Select configuration options, then run {\tt configure}, for example,
for the FVM package, on an account named users, with
a specific installation of the  MED IO library,
based on a system-wide installation of HDF5 (which will be auto-detected),
and using MPI compiler wrappers, a configure command may look like this:

\fbox{\begin{minipage}{\textwidth}\texttt{\\
\$ /home/user/Code\_Saturne/2.0/fvm-0.15.3/configure \textbackslash \\
\textcolor{Violet}{--prefix}=/home/user/Code\_Saturne/2.0/arch/prod \textbackslash \\
\textcolor{Magenta}{--with-bft}=/home/user/Code\_Saturne/2.0/arch/prod \textbackslash \\
\textcolor{Violet}{--with-med}=/home/user/opt/med-3.0 \textbackslash \\
\textcolor{red}{CC}=/home/user/opt/mpich2-1.4/bin/mpicc
}\end{minipage}}

We may note that as FVM requires BFT, the installation path
prefix of BFT must be provided using \texttt{--with-bft=}
if BFT is installed in a non-default system path. If BFT is installed
under \texttt{/usr} or \texttt{/usr/local} (which is the default
id no \texttt{--prefix=} option was provided to the BFT configuration,
it will be detected automatically, and the \texttt{--with-bft=}
does not need to be provided. The same holds true for the FVM
and MEI libraries when configuring the \CS kernel (NCS).

In the rest of this section, we will assume that we are in
a build directory separate from sources, as described in
\S\ref{sec:prepare}. In different examples, we assume
that third-party libraries used by \CS are either available
as part of the base system (i.e. as packages in a Linux distribution),
as Environment Modules, or are installed under a separate path.

\subsection{Debug builds\label{sec:config:debug}}

It may be useful to install debug builds alongside production
builds of \CS, especially when user subroutines are used
and the risk of crashes due to user programming error is high.
Running the code using a debug build is significantly
slower, but more information may be available in the case
of a crash, helping understand and fix the problem faster.

Here, having a consistent and practical naming scheme is useful.
For a side-by-side debug build for the example above, we simply replace \texttt{prod} by
\texttt{dbg} in the \texttt{--prefix} option, and add
\texttt{--enable-debug} to the configure command:

\fbox{\begin{minipage}{\textwidth}\texttt{\\
\$ /home/user/Code\_Saturne/2.0/fvm-0.15.3/configure \textbackslash \\
\textcolor{Violet}{--prefix}=/home/user/Code\_Saturne/2.0/arch/prod \textbackslash \\
\textcolor{Magenta}{--with-bft}=/home/user/Code\_Saturne/2.0/arch/\textcolor{Magenta}{dbg} \textbackslash \\
\textcolor{Violet}{--with-med}=/home/user/opt/med-3.0 \textbackslash \\
\textcolor{Magenta}{--enable-debug} \textbackslash \\
\textcolor{red}{CC}=/home/user/opt/mpich2-1.4/bin/mpicc
}\end{minipage}}

Debug and non debug packages may be mixed, so in this example,
we could use a non-debug build of BFT, bit for consistency
and better debugging capabilities, it is recommended to have
a full debug version alongside a production version.

\subsection{Shared or static builds\label{sec:config:shared}}

By default, on most architectures, both shared and static libraries
for \CS will be built, and the executables will be linked with shared
libraries by default. To disable either shared or static libraries,
add either {\tt --disable-shared} or {\tt --disable-static}
to the options passed to {\tt configure}. This will speed-up the build,
process as each file will only be built once, and not twice.

In some cases, a shared build may fail due to some dependencies
on static-only MPI libraries. In this case, {\tt --disable-shared}
will be necessary. Disabling shared libraries has also been seen
to avoid issues with linking on Mac OSX systems.

In any case, be careful if you switch from one option to the other: as
linking will be done with shared libraries by default, a build
with static libraries only will not completely overwrite a build using
shared libraries, so uninstalling the previous build first
is recommended.

\subsection{MPI compiler wrappers\label{sec:config:mpicc}}

MPI environments generally provide compiler wrappers, usually
with names similar to \texttt{mpicc} for C and \texttt{mpif90} for
Fortran 90. Wrappers conforming to the MPI standard recommendations should
provide a \texttt{-show} option, to show which flags are added to the compiler
so as to enable MPI. Using wrappers is fine as long as several third-party tools
do not provide their own wrappers, in which case either
a priority must be established. For example, using HDF5's
\texttt{h5pcc} compiler wrapper includes the options used by
\texttt{mpicc} when building HDF5 with parallel IO, in addition to
HDF5's own flags, so it could be used instead of \texttt{mpicc}.
On the contrary, when using a serial build of HDF5 for a parallel
build of \CS, the \texttt{h5cc} and \texttt{mpicc} wrappers
contain different flags, so they are in conflict.

Also, some MPI compiler wrappers may include optimization options
used to build MPI, which may be different from those we wish to use
that were passed. 

To avoid issues with MPI wrappers, it is possible to select an
MPI library using the \texttt{--with-mpi} option to \texttt{configure}.
For finer control, \texttt{--with-mpi-include} and \texttt{--with-mpi-lib}
may be defined separately.

Still, this only works in some cases, as a fixed list of libraries
is tested for, so using MPI compiler wrapper is the simplest and safest
solution. Simply use a \texttt{CC=mpicc} or similar option instead
of \texttt{--with-mpi}.\footnote{On Linux distributions such
as Debian or Ubuntu, \texttt{--with-mpi} works with Open MPI, but
\texttt{CC=mpicc} is necessary to build with MPICH2}.

MPI compiler wrappers only need to be used for the C compiler,
and only for FVM and for the kernel.
\emph{Never} use an \texttt{FC=mpif90}
or equivalent option: in \CS, MPI is never called directly from Fortran code,
so Fortran MPI bindings are not necessary, but they can lead to
build failures, especially in cross-compilation
configurations.\footnote{\texttt{configure} determines which libraries are
necessary to link the Fortran runtime using a C compiler as a linker.
When using an MPI fortran wrapper, extra libraries that are not normally
necessary will be added to those we link with, and the Libtool script
that is part of the build system will often try to add further dependencies,
mixing-up front-end and compute node compiler options and
libraries (Libtool may be very practical when it works, but in complex
situations where is guesses incorrectly at the commands it should run, it always
acts as if it knows best, and is very difficult to work around).}

\subsection{Environment Modules\label{sec:config:envmode}}

As noted in \S\ref{sec:obtain_ext_lib}, on systems providing
Environment Modules with the {\tt module} command, \CS's {\tt configure}
script detects which modules are loaded and saves
this list so that future runs of the code use that same environment,
rather than the user's environment, so as to allows using versions of
\CS built with different modules safely and easily.

Given this, it is recommended that when configuring and installing
\CS, only the modules necessary for that build be loaded.
Note that as \CS uses the module
environment detected and runtime instead of the user's current
module settings, debuggers requiring a specific module may
not work under a standard run script if they were not loaded when
installing the code.

\subsection{Compiler flags and environment variables\label{sec:config:flags}}

As usual when using an Autoconf-based \texttt{configure} script,
some environment variables may be used. \texttt{configure --help}
will provide the list of recognized variables.
\texttt{CC} and \texttt{FC} allow selecting the C and Fortran compiler
respectively (possibly using an MPI compiler wrapper for the C parts
of FVM and the Kernel).

Compiler options are usually defined automatically, based on
detection of the compiler (and depending on whether \texttt{--enable-debug}
was used). This is handled in a \texttt{config/<package\_name>\_auto\_flags.sh}
script in all of CS's packages (for example,
\texttt{config/cs\_\_auto\_flags.sh}
for the Kernel. This file is sourced when running \texttt{configure}, so
any modification to it will be effective as soon as \texttt{configure} is run.
When installing on an exotic machine, or with a new compiler, adapting this
file is useful (and providing feedback to the \CS development team
will enable support of a broader range of compilers and systems in the
future.

To disable automatic setting of flags, the \texttt{--disable-auto-flags}
option may be used, and the classical \texttt{CPPFLAGS}, \texttt{CFLAGS},
\texttt{FCCFLAGS}, \texttt{LDFLAGS}, and \texttt{LIBS} environment variables
may be used. A mix of automatic and prescribed flags may also be experimented.
for packages other than the Kernel, \texttt{CPP\_ADD}, \texttt{CC\_ADD},
\texttt{LD\_ADD}, and \texttt{LIBS\_ADD} may be used to add options without
overriding automatic flags. For the kernel, the standard flags provided by
the user are appended to the automatic flags (which may be disabled
if only user flags are desired).

\section{Compile and install\label{sec:compile}}

Once a package is configured, it may be compiled and installed;
for example, to compile a configured package (using 4 parallel threads),
then install it, run:

\texttt{\$ make -j 4 \&\& make install}

The solver (NCS) also includes the documentation, which may be built
with:

\texttt{\$ make pdf \&\& make install-pdf}

To clean the build directory, keeping the configuration,
use \texttt{make clean};
To uninstall an installed build, use \texttt{make uninstall}.
To clear all configuration info, use \texttt{make distclean}
(\texttt{make uninstall} will not work after this).

\section{Remarks for very large meshes\label{sec:config:largemesh}}

If \CS is to be run on large meshes, several precautions regarding
its configuration and that of third-party software must be taken.

in addition to local connectivity arrays, \CS uses global element ids
for some operations, such as reading and writing meshes and restart files,
parallel interface element matching, and post-processing output.
For a hexahedral mesh with $N$ cells,
the number of faces is about $3N$ (6 faces per cell, shared by 2 cells each).
With 4 cells per face, the $face \rightarrow vertices$ array is of size
of the order of $4\times3N$, so global ids used in that array's index
will reach $2^{31}$ for a mesh in the range of $2^{31} / 12 \approxeq 178.10^6$.
In practice, we have encountered a limit with slightly smaller meshes,
around 150 million cells.

Above 150 million hexahedral cells or so, it is thus imperative to configure
the build to use 64-bit global element ids, with the
{\tt --enable-long-gnum} option. Local indexes will still use
the default {int} size, so memory consumption will only be slightly
increased.

Recent versions of some third-party libraries may also optionally use 64-bit ids,
independently of each other or of \CS.
This is the case for the \scotch and \metis, MED and
CGNS libraries. In the case of graph-based partitioning, only
global cell ids are used, so 64-bit ids should not in theory be necessary
for meshes under 2 billion cells. In a similar vein, for post-processing output
using nodal connectivity, 64-bit global ids should only be an imperative
when the number of cells or vertices approaches 2 billion.
Practical limits may be lower, if some intermediate internal counts
reach these limits earlier.

Note also that \metis 4 is known to crash for meshes in the range of
35 million cells and above, so \metis 5 or \scotch are necessary.
Partitioning a 158 million hexahedral mesh using \metis 5
on a front-end node with 128 Gb memory is possible,
but partitioning the same mesh on cluster nodes with 24 Gb each
may not, so using front-end nodes with a large amount of memory may be
necessary to optimize partitioning of large meshes.
The backup solution is to simply use the built-in parallel Morton Z curve
partitioning, which may produce partitions of lower quality, but
runs transparently at \CS calculation initialization, and is highly
scalable.


\section{Troubleshooting\label{sec:config:troubleshoot}}

If \texttt{configure} fails and reports an error, the message should
be sufficiently clear in most case to understand the cause of the
error and fix it. Do not forget that for libraries installed using
packages, the development versions of those packages are also
necessary, so if configure fails to detect a package which you
believe is installed, check the matching development package.

Also, whether it succeeds or fails, \texttt{configure} generates
a file named \texttt{config.log}, which contains details on tests
run by the script, and is very useful to troubleshoot
configuration problems. When \texttt{configure} fails due to a given
third-party library, details on tests relative to that library
are found in the \texttt{config.log} file. The interesting information
is usually in the middle of the file, so you will need to search
for strings related to the library to find the test that failed
and detailed reasons for its failure.

\section{Installing for \syrthes coupling\label{sec:syrthes}}

When coupling with \syrthes using MPI, both \CS and \syrthes must
use the same MPI library. Coupling with \syrthes is also possible
using TCP/IP sockets, but this is rarely used, and is recommended only
when running on a single node. On compute clusters, the MPI environment
will distribute \syrthes and \CS instances to different processors when
they are coupled using MPI, while using sockets, the \syrthes process will
usually run on a node on which \CS is also running, which may degrade
performance. On a single node, this is not an issue, and allows coupling
\CS and \syrthes even with serial builds.
In any case, the \texttt{--with-syrthes=} option of the
Kernel's \texttt{configure} script must be used so that
\CS may find \syrthes.

\section{Example configuration and build commands\label{sec:config:examples}}

Most available prerequisites are auto-detected, so to install the
code to the default \texttt{/usr/local} sub-directory,
a command such as:

\texttt{\$ ../../saturne/configure}

should be sufficient.

For the following examples, Let us define environment variables respectively
reflecting the \CS source path, installation path, and a path where optional
libraries are installed:

\fbox{\begin{minipage}{\textwidth}\texttt{\\
\$ \textcolor{OliveGreen}{SRC\_PATH}=/home/projects/Code\_Saturne/2.0/src \\
\$ \textcolor{OliveGreen}{INSTALL\_PATH}=/home/projects/Code\_Saturne/2.0 \\
\$ \textcolor{OliveGreen}{CS\_OPT}=/home/projects/opt
}\end{minipage}}

For an install on which multiple
versions and architectures of the code should be available,
configure commands with all bells and whistles (except SALOME support) for a
build on a cluster named \texttt{ivanoe}, using the Intel compilers
(made available through environment modules) is described in the following
examples.

First, environment modules are loaded if necessary:

\fbox{\begin{minipage}{\textwidth}\texttt{\\
\$ \textcolor{Magenta}{module purge} \\
\$ \textcolor{Magenta}{module load} intel\_compilers/12.0.3.174 \\
\$ \textcolor{Magenta}{module load} open\_mpi/gcc/1.4.3
}\end{minipage}}

\subsection{BFT build\label{sec:bft:config:examples}}

To configure and install BFT, we use the following commands:

\fbox{\begin{minipage}{\textwidth}\texttt{\\
\$ mkdir -p bft-1.0.5\_build/ivanoe \&\& cd bft-1.0.5\_build/ivanoe\\
\$ \textcolor{OliveGreen}{\$SRC\_PATH}/bft-1.0.5/configure \textbackslash \\
\textcolor{Violet}{--prefix}=\textcolor{OliveGreen}{\$INSTALL\_PATH}/arch/ivanoe \textbackslash \\
\textcolor{red}{CC}=icc \\
\$ make -j 4 \&\& make install \&\& make clean\\
\$ cd ..
}\end{minipage}}

\subsection{Preprocessor build\label{sec:ecs:config:examples}}

Once BFT is installed, to configure and install the Preprocessor, we may
use the following commands:

\fbox{\begin{minipage}{\textwidth}\texttt{\\
\$ mkdir -p ecs-2.0.2\_build/ivanoe \&\& cd ecs-2.0.2\_build/ivanoe\\
\$ \textcolor{OliveGreen}{\$SRC\_PATH}/ecs-2.0.2/configure \textbackslash \\
\textcolor{Violet}{--prefix}=\textcolor{OliveGreen}{\$INSTALL\_PATH}/arch/ivanoe
\textbackslash \\
\textcolor{Magenta}{--with-bft}=\textcolor{OliveGreen}{\$INSTALL\_PATH}/arch/ivanoe \\
\textcolor{Violet}{--with-hdf5}=\textcolor{OliveGreen}{\$CS\_OPT}/hdf5-1.8.7/arch/ivanoe
\textbackslash \\
\textcolor{Violet}{--with-med}=\textcolor{OliveGreen}{\$CS\_OPT}/med-3.0/arch/ivanoe
\textbackslash \\
\textcolor{Violet}{--with-cgns}=\textcolor{OliveGreen}{\$CS\_OPT}/cgns-3.1/arch/ivanoe \textbackslash \\
\textcolor{Violet}{--with-ccm}=\textcolor{OliveGreen}{\$CS\_OPT}/libccmio-2.6.19/arch/ivanoe \textbackslash \\
\textcolor{Violet}{--with-scotch}=\textcolor{OliveGreen}{\$CS\_OPT}/scotch-5.1.11/arch/ivanoe \textbackslash \\
\textcolor{Violet}{--with-metis}=\textcolor{OliveGreen}{\$CS\_OPT}/parmetis-3.2/arch/ivanoe \textbackslash \\
\textcolor{red}{CC}=icc \\
\$ make -j 4 \&\& make install \&\& make clean\\
\$ cd ..
}\end{minipage}}

\subsection{FVM build\label{sec:fvm:config:examples}}

Once BFT is installed, to configure and install FVM, we may use the
following commands:

\fbox{\begin{minipage}{\textwidth}\texttt{\\
\$ mkdir -p fvm-0.15.3\_build/ivanoe \&\& cd fvm-0.15.3\_build/ivanoe\\
\$ \textcolor{OliveGreen}{\$SRC\_PATH}/fvm-0.15.3/configure \textbackslash \\
\textcolor{Violet}{--prefix}=\textcolor{OliveGreen}{\$INSTALL\_PATH}/arch/ivanoe\_ompi
\textbackslash \\
\textcolor{Magenta}{--with-bft}=\textcolor{OliveGreen}{\$INSTALL\_PATH}/arch/ivanoe
\textbackslash \\
\textcolor{Violet}{--with-hdf5}=\textcolor{OliveGreen}{\$CS\_OPT}/hdf5-1.8.6/arch/ivanoe\_ompi
\textbackslash \\
\textcolor{Violet}{--with-med}=\textcolor{OliveGreen}{\$CS\_OPT}/med-3.0/arch/ivanoe\_ompi
\textbackslash \\
\textcolor{Violet}{--with-cgns}=\textcolor{OliveGreen}{\$CS\_OPT}/cgns-3.1/arch/ivanoe\_ompi \textbackslash \\
\textcolor{red}{CC}=mpicc \\
\$ make -j 4 \&\& make install \&\& make clean\\
\$ cd ..
}\end{minipage}}

In the example above, we have appended the \texttt{\_ompi} postfix
to the architecture name, in case we intend to install 2 builds,
with different MPI libraries (such as Open MPI and MPICH2).
Note that optional libraries using MPI must also use the same MPI
library. This is the case for HDF5, CGNS, and MED if they are built
with MPI-IO support (which is not currently used by \CS for these libraries).
Similarly, C++ and Fortran libraries, and even C libraries built with recent
optimizing C compilers, may require runtime libraries associated to that
compiler, so if versions using different compilers are to be installed, it is
recommended to use a naming scheme which reflects this.
In this example, HDF5, CGNS and MED were built without MPI-IO support,
as FVM (and thus \CS) does not yet exloit MPI-IO for these libraries.

\subsection{MEI build\label{sec:mei:config:examples}}

Once BFT is installed, to configure and install MEI, we may
use the following commands:

\fbox{\begin{minipage}{\textwidth}\texttt{\\
\$ mkdir -p mei-1.0.2\_build/ivanoe \&\& cd mei-1.0.2\_build/ivanoe\\
\$ \textcolor{OliveGreen}{\$SRC\_PATH}/mei-1.0.2/configure \textbackslash \\
\textcolor{Violet}{--prefix}=\textcolor{OliveGreen}{\$INSTALL\_PATH}/arch/ivanoe
\textbackslash \\
\textcolor{Magenta}{--with-bft}=\textcolor{OliveGreen}{\$INSTALL\_PATH}/arch/ivanoe
\textbackslash \\
\textcolor{Violet}{--with-python-exec}=\textcolor{OliveGreen}{\$CS\_OPT}/python-2.7.1/arch/ompi/bin \textbackslash \\
\textcolor{Violet}{--with-swig-exec}=\textcolor{OliveGreen}{\$CS\_OPT}/swig-1.3.39/arch/ompi/bin \textbackslash \\
\textcolor{red}{CC}=icc
\$ make -j 4 \&\& make install \&\& make clean\\
\$ cd ..
}\end{minipage}}

\subsubsection{Using a specific Python interpreter%
\label{sec:mei:config:examples:python}}

Note that the MEI's \texttt{--with-python-exec=} option allows specifying
the path where the \texttt{python} executable will be found, but
not the name of the executable. On Linux distributions where several
Python interpreters are installed under \texttt{/usr/bin} using
different names, such as \texttt{python}, \texttt{python2.6},
and \texttt{python2.7}, only the default may be used.
If necessary, a workaround is to create a wrapper script named
\texttt{python} in a directory named \texttt{bin}, which
calls the desired Python interpreter and passes arguments.
Using the installation's target directory seems a natural
solution, for example:

\fbox{\begin{minipage}{\textwidth}\texttt{\\
\$ cd \textcolor{OliveGreen}{\$INSTALL\_PATH}/arch/ivanoe/bin\\
\$ echo "\#/bin/sh" > python \\
\$ echo "/usr/bin/python2.7 \$@" >> python \\
\$ chmod +x python \\
\$ cd -
}\end{minipage}}

The \texttt{bin} directory containing the \texttt{python} wrapper
may now be passed to the \texttt{--with-python-exec=}
\texttt{configure} option.

\subsection{Solver build\label{sec:bft:config:examples}}

\fbox{\begin{minipage}{\textwidth}\texttt{\\
\$  mkdir -p ncs-2.0.2\_build/ivanoe \&\& cd ncs-2.0.2\_build/ivanoe\\
\$ \textcolor{OliveGreen}{\$SRC\_PATH}/ncs-2.0.2/configure \textbackslash \\
\textcolor{Violet}{--prefix}=\textcolor{OliveGreen}{\$INSTALL\_PATH}/arch/ivanoe\_ompi
\textbackslash \\
\textcolor{Magenta}{--with-bft}=\textcolor{OliveGreen}{\$INSTALL\_PATH}/arch/ivanoe \textbackslash\\
\textcolor{Magenta}{--with-fvm}=\textcolor{OliveGreen}{\$INSTALL\_PATH}/arch/ivanoe\_ompi
\textbackslash \\
\textcolor{Magenta}{--with-mei}=\textcolor{OliveGreen}{\$INSTALL\_PATH}/arch/ivanoe
\textbackslash \\
\textcolor{Magenta}{--with-prepro}=\textcolor{OliveGreen}{\$INSTALL\_PATH}/arch/ivanoe \textbackslash \\
\textcolor{Violet}{--with-blas}=/opt/intel/composerxe-2011.3.174/mkl \textbackslash \\
\textcolor{Violet}{--with-libxml2}=\textcolor{OliveGreen}{\$CS\_OPT}/libxml2-2.3.32/arch/ivanoe \textbackslash \\
\textcolor{Violet}{--with-syrthes}=/home/projects/syrthes3.4.3 \textbackslash \\
\textcolor{Violet}{--with-python-exec}=\textcolor{OliveGreen}{\$CS\_OPT}/python-2.7.1/arch/ompi/bin \textbackslash \\
\textcolor{red}{CC}=mpicc \textcolor{red}{FC}=ifort
\$  make -j 4 \&\& make install \&\& make clean\\
\$ cd ..
}\end{minipage}}

The remarks regarding the selection of a Python interpreter
as in \S\ref{sec:mei:config:examples:python} hold here also.

\subsection{Cross-compiling}

On machines with different front-end and compute node architectures,
such as IBM Blue Gene/P, cross-compiling is necessary;
some parts of the code need to be built for the front-end, other parts need
to be built for the compute nodes, and some parts need to be built for both:

\begin{itemize}
\item BFT will be needed both by the \pcs and the Kernel, so it must
      be built for both architectures.
\item The \pcs only needs to be built for the front-end.
\item If used, MEI needs to be built for the compute nodes,
      but a build on the front-end with Python bindings will be necessary
      for the GUI to allow defining MEI expressions.
\item FVM and NCS (the Kernel) only need to be built for the compute nodes,
      though a build for the front-end may be useful for mesh verification.
      In any case, the Python interpreter used by the GUI and scripts
      will only need to run on the front-end or service nodes:
      execution on the compute nodes is limited to the \texttt{cs\_solver}
      executable when launched using \texttt{mpirun}.
      using 
\end{itemize}

A debug variant of the compute node packages is also recommended, as always.
Providing a debug variant of the front-end packages is not generally useful.

Depending on their role, optional third-party libraries should be installed
either for the front-end, for the compute nodes, or both:

\begin{itemize}
\item BLAS will be useful only for the compute nodes, and are generally
      always available on large compute facilities.
\item Python and PyQt4 will run on the front-end node only.
\item Libxml2 must be available for the compute nodes if the GUI is used.
\item HDF5, MED and CGNSlib may be used by the Preprocessor on the front-end node
      to import meshes, and by the main solver on the compute nodes
      to output visualization meshes and fields.
\item libCCMIO is used by the Preprocessor exclusively, so it may be needed on the
      front-end node only.
\item \scotch or \metis may be used by a front-end node build of the
      partitioner, as serial partitioning of large meshes requires a lot of memory.
\end{itemize}

In the following subsections, only the \texttt{configure} commands are
detailed, as the creation of build directories and the running of
\texttt{make} and \texttt{make install} commands has already
been described for other cases, and bring nothing new here.

In our example, the front-end node is based on an IBM Power architecture,
on which the GCC compiler is available, but produces 32-bit code by default.
Adding the \texttt{"-m64"} flags force the compiler into 64-bit mode, allowing
the \pcs to import meshes up into the 100-million cell range.

An additional complexity stems from this 32-bit default, as we have
had difficulty forcing the build of Qt4 in 64-bit mode on such
a front-end node running SUSE Linux Enterprise Server 10, on which
it is not pre-installed. Forcing 64-bit mode leads to link errors,
so we use a 32-bit version. This implies that we must also use a
32-bit build of Python for PyQt4, which in turn means that the Python
bindings for MEI used by the GUI must be 32-bit code,
also requiring an additional build of BFT.

\subsubsection{BFT configurations}

For the compute nodes:

\fbox{\begin{minipage}{\textwidth}\texttt{\\
\$ \textcolor{OliveGreen}{\$SRC\_PATH}/bft-1.0.5/configure \textbackslash \\
\textcolor{Violet}{--prefix}=\textcolor{OliveGreen}{\$INSTALL\_PATH}/arch/bgp \textbackslash \\
--without-zlib \textbackslash \\
\textcolor{Magenta}{--build}=ppc64 \textbackslash \\
\textcolor{Magenta}{--host}=bluegenep \textbackslash \\
\textcolor{red}{CC}=bgxlc
}\end{minipage}}

Here, the \texttt{--build=ppc64 --host=bluegenep} options ensure the
\texttt{configure} script is forced into cross-compilation mode.
With a front-end base on an Intel or AMD, architecture,
\texttt{--build=x86\_64} or \texttt{--build=amd64} should be used
instead of \texttt{--build=ppc64}. For the host (target) architecture,
\texttt{--host=bluegenep} is recognized by current GNU Autoconf versions,
so it is preferred, but \texttt{--host=ppc} also works well. Actually,
any choice of build and host architectures recognized by Autoconf would
probably work, as long as build and host are different.

For the front-end nodes:

\fbox{\begin{minipage}{\textwidth}\texttt{\\
\$ \textcolor{OliveGreen}{\$SRC\_PATH}/bft-1.0.5/configure \textbackslash \\
\textcolor{Violet}{--prefix}=\textcolor{OliveGreen}{\$INSTALL\_PATH}/arch/frontend \textbackslash \\
\textcolor{red}{CC\_ADD}="-m64" \textbackslash \\
\textcolor{red}{LD\_ADD}="-m64"
}\end{minipage}}

For the Python bindings on the front-end nodes:

\fbox{\begin{minipage}{\textwidth}\texttt{\\
\$ \textcolor{OliveGreen}{\$SRC\_PATH}/bft-1.0.5/configure \textbackslash \\
\textcolor{Violet}{--prefix}=\textcolor{OliveGreen}{\$INSTALL\_PATH}/arch/frontend\_python
}\end{minipage}}

\subsubsection{\pcs configuration}

The \pcs only needs to be built for the front-end nodes, so apart
from forcing the compiler in 64-bit mode, its configuration is classical:

\fbox{\begin{minipage}{\textwidth}\texttt{\\
\$ \textcolor{OliveGreen}{\$SRC\_PATH}/ecs-2.0.2/configure \textbackslash \\
\textcolor{Violet}{--prefix}=\textcolor{OliveGreen}{\$INSTALL\_PATH}/arch/frontend \textbackslash \\
\textcolor{Magenta}{--with-bft}=\textcolor{OliveGreen}{\$INSTALL\_PATH}/arch/frontend \textbackslash \\
\textcolor{Violet}{--with-hdf5}=\textcolor{OliveGreen}{\$CS\_OPT}/hdf5-1.8.7/arch/frontend
\textbackslash \\
\textcolor{Violet}{--with-med}=\textcolor{OliveGreen}{\$CS\_OPT}/med-3.0/arch/frontend
\textbackslash \\
\textcolor{Violet}{--with-cgns}=\textcolor{OliveGreen}{\$CS\_OPT}/cgns-3.1/arch/frontend \textbackslash \\
\textcolor{Violet}{--with-ccm}=\textcolor{OliveGreen}{\$CS\_OPT}/libccmio-2.6.1/arch/frontend \textbackslash \\
\textcolor{Violet}{--with-scotch}=\textcolor{OliveGreen}{\$CS\_OPT}/scotch-5.1.11/arch/frontend \textbackslash \\
\textcolor{Violet}{--with-metis}=\textcolor{OliveGreen}{\$CS\_OPT}/metis-5.0rc1/arch/frontend \textbackslash \\
\textcolor{red}{CC\_ADD}="-m64" \textbackslash \\
\textcolor{red}{LD\_ADD}="-m64"
}\end{minipage}}

\subsubsection{FVM configuration}

For the compute nodes, the following FVM configuration may be
used if no MED of CGNS output is necessary:

\fbox{\begin{minipage}{\textwidth}\texttt{\\
\$ \textcolor{OliveGreen}{\$SRC\_PATH}/fvm-0.15.3/configure \textbackslash \\
\textcolor{Violet}{--prefix}=\textcolor{OliveGreen}{\$INSTALL\_PATH}/arch/bgp \textbackslash \\
\textcolor{Magenta}{--with-bft}=\textcolor{OliveGreen}{\$INSTALL\_PATH}/arch/bgp \textbackslash \\
\textcolor{Magenta}{--build}=ppc64 \textbackslash \\
\textcolor{Magenta}{--host}=bluegenep \textbackslash \\
\textcolor{red}{CC}=mpixlc
}\end{minipage}}

CGNS support may be configured normally, but FVM output is a bit more
involved: MED uses C++ runtime libraries, which need to be defined explicitly
for the code to link correctly. The \texttt{--with-med-dep-dirs}
and \texttt{--with-med-dep-libs} options, which may define comma-separated
values (no whitespace) allow defining this and passing this info to
the Kernel build, but an additional complication is that \texttt{libstdc++}
should not be defined through a \texttt{-lstdc++} option, but through
a absolute path: otherwise, Libtool tends to confuse front-end and compute
node libraries, and add front-end library options to the compute-node
link stage, leading to an error. There is no clean way around this,
so we pass the absolute path to that library using the \texttt{LIBS}
variable:

\fbox{\begin{minipage}{\textwidth}\texttt{\\
\$ \textcolor{OliveGreen}{\$SRC\_PATH}/fvm-0.15.3/configure \textbackslash \\
\textcolor{Violet}{--prefix}=\textcolor{OliveGreen}{\$INSTALL\_PATH}/arch/bgp \textbackslash \\
\textcolor{Magenta}{--with-bft}=\textcolor{OliveGreen}{\$INSTALL\_PATH}/arch/bgp \textbackslash \\
\textcolor{Violet}{--with-hdf5}=\textcolor{OliveGreen}{\$CS\_OPT}/hdf5-1.8.6/arch/bgp
\textbackslash \\
\textcolor{Violet}{--with-med}=\textcolor{OliveGreen}{\$CS\_OPT}/med-3.0/arch/bgp
\textbackslash \\
\textcolor{Violet}{--with-med-dep-dirs}=/opt/ibmcmp/vacpp/bg/9.0/bglib \textbackslash \\
\textcolor{Violet}{--with-med-dep-libs}=ibmc++ \textbackslash \\
\textcolor{Violet}{--with-cgns}=\textcolor{OliveGreen}{\$CS\_OPT}/cgns-3.1/arch/bgp \textbackslash \\
\textcolor{Magenta}{--build}=ppc64 \textbackslash \\
\textcolor{Magenta}{--host}=bluegenep \textbackslash \\
\textcolor{red}{LIBS}=/bgsys/drivers/ppcfloor/ppc/gnu-linux/powerpc-bgp-linux/lib/libstdc++.a \textbackslash \\
\textcolor{red}{CC}=mpixlc
}\end{minipage}}

Note that MED is generally more useful for the \pcs than for post-processing:
meshes built using the SALOME tool will be produced in MED format, so
importing it may be useful. For post-processing, other formats such
as the \ensight format are also handled by ParaView and VisIt, so
MED is only useful if required by a downstream calculation.
In addition, MED 2.3 requires arrays to be output in one piece, and
the parallel IO enabled by the recent MED 3 releases are not yet
handled by FVM. This means that when writing mesh vertices, a global
array containing coordinates of all vertices from all processors
must be gathered to the first processor, which may require more memory
than available on the compute nodes even for mid-sized meshes
(as CGNS has allowed partial writes and \ensight output is handled
internally, we only have this limitation with MED output).
MED output may be useful for example to output a boundary mesh
with pressure data usable by a solid mechanics code such as
\emph{Code\_Aster}, but it will not be usable for the main volume
for the large meshes we may expect to use on Blue Gene type machines.
In addition, linking with MED and HDF5 significantly increase
the size of the \CS executable, bringing it above 50 Mb,
so building FVM with support for MED output may be worthwhile
only if the use of this format is truly required.

A front-end configuration is more standard:

\fbox{\begin{minipage}{\textwidth}\texttt{\\
\$ \textcolor{OliveGreen}{\$SRC\_PATH}/fvm-0.15.3/configure \textbackslash \\
\textcolor{Violet}{--prefix}=\textcolor{OliveGreen}{\$INSTALL\_PATH}/arch/frontend \textbackslash \\
\textcolor{Magenta}{--with-bft}=\textcolor{OliveGreen}{\$INSTALL\_PATH}/arch/frontend \textbackslash \\
\textcolor{Violet}{--with-hdf5}=\textcolor{OliveGreen}{\$CS\_OPT}/hdf5-1.8.6/arch/frontend \textbackslash \\
\textcolor{Violet}{--with-med}=\textcolor{OliveGreen}{\$CS\_OPT}/med-3.0/arch/frontend
\textbackslash \\
\textcolor{red}{CC\_ADD}="-m64" \textbackslash \\
\textcolor{red}{LD\_ADD}="-m64"
}\end{minipage}}

\subsubsection{MEI configuration}

For the compute nodes, Python bindings are not useful:

\fbox{\begin{minipage}{\textwidth}\texttt{\\
\$ \textcolor{OliveGreen}{\$SRC\_PATH}/mei-1.0.2/configure \textbackslash \\
\textcolor{Violet}{--prefix}=\textcolor{OliveGreen}{\$INSTALL\_PATH}/arch/bgp \textbackslash \\
\textcolor{Magenta}{--with-bft}=\textcolor{OliveGreen}{\$INSTALL\_PATH}/arch/bgp \textbackslash \\
\textcolor{Violet}{--disable-python-bindings} \textbackslash\\
\textcolor{Magenta}{--build}=ppc64 \textbackslash \\
\textcolor{Magenta}{--host}=bluegenep \textbackslash \\
\textcolor{red}{CC}=bgxlc
}\end{minipage}}

For the front-end nodes, only the Python bindings are generally useful. The
kernel build should only be used for mesh verification (if built at
all), so it will not need evaluation of mathematical expressions:

\fbox{\begin{minipage}{\textwidth}\texttt{\\
\$ \textcolor{OliveGreen}{\$SRC\_PATH}/mei-1.0.2/configure \textbackslash \\
\textcolor{Violet}{--prefix}=\textcolor{OliveGreen}{\$INSTALL\_PATH}/arch/bgp \textbackslash \\
\textcolor{Magenta}{--with-bft}=\textcolor{OliveGreen}{\$INSTALL\_PATH}/arch/bgp \textbackslash \\
\textcolor{Violet}{--with-python-exec}=\textcolor{OliveGreen}{\$CS\_OPT}/python-2.7.1/arch/frontend/bin \textbackslash \\
\textcolor{Violet}{--with-swig-exec}=\textcolor{OliveGreen}{\$CS\_OPT}/swig-1.3.39/arch/frontend/bin \textbackslash
}\end{minipage}}

\subsubsection{Kernel configuration}

Finally, when all other packages are installed, the Kernel may
be configured. For the compute nodes, the following command may be used:

\fbox{\begin{minipage}{\textwidth}\texttt{\\
\$ \textcolor{OliveGreen}{\$SRC\_PATH}/ncs-2.0.2/configure \textbackslash \\
\textcolor{Violet}{--prefix}=\textcolor{OliveGreen}{\$INSTALL\_PATH}/arch/bgp\_ompi
\textbackslash \\
\textcolor{Magenta}{--with-bft}=\textcolor{OliveGreen}{\$INSTALL\_PATH}/arch/bgp \textbackslash \\
\textcolor{Magenta}{--with-fvm}=\textcolor{OliveGreen}{\$INSTALL\_PATH}/arch/bgp \textbackslash \\
\textcolor{Magenta}{--with-mei}=\textcolor{OliveGreen}{\$INSTALL\_PATH}/arch/bgp \textbackslash \\
\textcolor{Magenta}{--with-prepro}=\textcolor{OliveGreen}{\$INSTALL\_PATH}/arch/frontend \textbackslash \\
\textcolor{Violet}{--with-blas}=/opt/ibmmath/essl/4.4 \textbackslash \\
\textcolor{Violet}{--with-libxml2}=\textcolor{OliveGreen}{\$CS\_OPT}/libxml2-2.3.32/arch/frontend \textbackslash \\
\textcolor{Violet}{--with-syrthes}=/home/projects/syrthes3.4.3 \textbackslash \\
\textcolor{Violet}{--disable-sockets} \textbackslash \\
\textcolor{Violet}{--disable-dlloader} \textbackslash \\
\textcolor{Violet}{--disable-nls} \textbackslash \\
\textcolor{Magenta}{--build}=ppc64 \textbackslash \\
\textcolor{Magenta}{--host}=bluegenep \textbackslash \\
\textcolor{Violet}{--with-python-exec}=\textcolor{OliveGreen}{\$CS\_OPT}/python-2.7.1/arch/frontend/bin \textbackslash \\
\textcolor{red}{CC}=mpixlc\_r \textcolor{red}{FC}=bgxlf90\_r
}\end{minipage}}

The thread-safe compiler wrappers used here should not be necessary for \CS, but in our
experience, the ESSL BLAS are correctly detected only with those wrappers,
not with the single-threaded versions.\footnote{This might be due to a bug in the ESSL
BLAS detection of \CS, although the code has been checked.}

Note that the front-end version of the \pcs is used even for
the compute node build, as only the Kernel itself will run on the
latter.
For the front-end nodes, the following command may be used:

\fbox{\begin{minipage}{\textwidth}\texttt{\\
\$ \textcolor{OliveGreen}{\$SRC\_PATH}/ncs-2.0.2/configure \textbackslash \\
\textcolor{Violet}{--prefix}=\textcolor{OliveGreen}{\$INSTALL\_PATH}/arch/bgp\_ompi
\textbackslash \\
\textcolor{Magenta}{--with-bft}=\textcolor{OliveGreen}{\$INSTALL\_PATH}/arch/bgp \textbackslash \\
\textcolor{Magenta}{--with-fvm}=\textcolor{OliveGreen}{\$INSTALL\_PATH}/arch/bgp \textbackslash \\
\textcolor{Magenta}{--without-mei} \textbackslash \\
\textcolor{Magenta}{--with-prepro}=\textcolor{OliveGreen}{\$INSTALL\_PATH}/arch/frontend \textbackslash \\
\textcolor{Violet}{--with-python-exec}=\textcolor{OliveGreen}{\$CS\_OPT}/python-2.7.1/arch/frontend/bin \textbackslash \\
\textcolor{red}{CFLAGS}="-m64" \textbackslash \\
\textcolor{red}{FCFLAGS}="-m64" \textbackslash \\
\textcolor{red}{LDFLAGS}="-m64"
}\end{minipage}}

\end{document}

